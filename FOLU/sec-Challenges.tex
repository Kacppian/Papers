


\section{First-Order Challenges}\label{sec:Challenges}

%TODO by Jan (just writing some ideas so far--not yet final by any means)\\
%{\bf todo: define the notion of set of literals resolved against a unit? might make things clearer}

In this section, we discuss the challenges introduced by adapting {\LowerUnits} to the first-order case. The first example illustrates how to extend {\LowerUnits} to first-order logic in the obvious way. Examples \ref{ex:pairwise} and on illustrate concerns that are introduced by the unification process that must be over come in order to successfully postpone resolution with a unit clause as a result of this extension.


%example 1: shows requirement for pair-wise unifiability with unit
 \begin{example}\label{ex:unif} Resolution with a unit clause $u$, with literal $\ell$, may be performed with a node $v$ provided $\dual{\ell} \in \Gamma_v$ and there exists a unifier $\sigma$ such that $\ell\sigma=\dual{\ell}$ (or $\ell=\dual{\ell}\sigma$). After applying $\sigma$ to the premises, the literals match syntactically, and this behaves like the propositional case. Thus the notion of looking for unifiable formulas to postpone resolution with $u$ is natural. Consider the following proof of $\psi$, where resolution with $\eta_2$ will be postponed:

\begin{footnotesize}
\begin{prooftree}
\def\e{\mbox{\ $\vdash$\ }}
\AxiomC{$\eta_1$: $p(Y)$\e$q(Z)$}
\AxiomC{$\eta_2$: \e$p(Y)$}
\BinaryInfC{$\eta_3$: \e$q(Z)$}
\AxiomC{$\eta_4$: $p(X),q(Z)$\e}
\BinaryInfC{$\eta_5$: $p(X)$\e}
\AxiomC{$\eta_2$}
\BinaryInfC{$\psi$: $\bot$}
\end{prooftree}
\end{footnotesize}

After postponing resolution with $\eta_2$, the formulas $p(Y)$ (in $\eta_1$) and $p(X)$ (in $\eta_4$) will both remain as in $\eta_3'$ below. Unlike the propositional case, where repeated literals would be dropped, in order to compress the proof, we must first contract these literals into a single literal $p(X)$ ($\eta_4'$). Reintroduction of $u$ completes the proof via resolution with the contracted literal.
\begin{footnotesize}
\begin{prooftree}
\def\e{\mbox{\ $\vdash$\ }}
\AxiomC{$\eta_1'$: $p(Y)$\e$q(Z)$}
\AxiomC{$\eta_2'$: $p(X),q(Z)$\e}
\BinaryInfC{$\eta_3'$: $p(X),p(Y)$\e}
\UnaryInfC{$\eta_4'$: $p(X)$\e}
\AxiomC{$\eta_5'$: \e$p(Y)$}
\BinaryInfC{$\psi$: $\bot$}
\end{prooftree}
\end{footnotesize}
 \end{example}


%example 2: shows requirement for pair-wise unifiability within all aux formulas

 \begin{example}\label{ex:pairwise} %The following example shows why we must check pair-wise unifiability with the literals resolved against the unit we're trying to lower. 
%When attempting to lower a unit clause in the first order case, additional properties must be satisfied. In addition to requiring that each resolved literal is unifiable with the unit literal, we require that all such unifiers between $u$ behave similarly in some sense.


Given that a contraction is required to compress the proof, we require that all literals $\dual{\ell}$ be simultaneously unifiable.
Consider the example below, where resolution with $u=\eta_2$ in the proof of $\psi$ will be postponed. 


\begin{footnotesize}
\begin{prooftree}
\def\e{\mbox{\ $\vdash$\ }}
\AxiomC{$\eta_2$}
\AxiomC{$\eta_4$: $r(X),p(b)$\e $s(Y)$}
\AxiomC{$\eta_1$: $p(a)$\e$q(Y),r(Z)$}
\AxiomC{$\eta_2$: \e $p(X)$}
\BinaryInfC{$\eta_3$: \e$q(Y),r(Z)$}
\BinaryInfC{$\eta_5$: $p(b)$\e $s(Y),q(Y)$}
\AxiomC{$\eta_6$: $s(Y), q(Y)$\e}
\insertBetweenHyps{\hskip -0.5in}
\BinaryInfC{$\eta_7$: $p(b)$\e}
\insertBetweenHyps{\hskip -0.8in}
\BinaryInfC{$\psi$: $\bot$}
\end{prooftree}
\end{footnotesize}

The literals resolved with $u$ are $p(a)$ (in $\eta_1$) and $p(b)$ (in $\eta_7$). If we attempt to postpone resolution, at the contraction step the clause would be $p(a),p(b)\vdash$; this clause cannot be contracted. Thus it would be a waste of time to attempt to postpone resolution with $u$, and so we require any unit we wish to lower to satisfy this property, which we define formally below.
 \end{example}

\begin{property}
\label{prop:pair}
%Things must be pair-wise unifiable... TODO: this? Or leave this out and talk about it section 5?
%Let $u$ be a unit clause with literal $\ell$, let $\dual{\ell}_1,\ldots,\dual{\ell}_n$ contained in clauses $v_1,\ldots,v_n$ be the literals that are unified with $\ell$ during resolution between $v_i$ and $u$ in the original proof. Then for every $\dual{\ell}_i$ and  $\dual{\ell}_j$ for $i\neq j$, there should be a unifier $\sigma_{i,j}$ such that after applying $\sigma_{i,j}$, $\dual{\ell}_i$ and $\dual{\ell}_j$ are syntactically equal.
Let $u$ be a unit clause with literal $\ell$, and let $\dual{\ell}_1 \in \Gamma_{v_1},\ldots,\dual{\ell}_n \in \Gamma_{v_n}$ be the literals that are unified with $\ell$ during resolution between $v_i$ and $u$ in a proof $\psi$. Then there should be a unifier $\sigma$ such that for every $\dual{\ell}_i$, there exists a unifier $\sigma_{i}$ such that $\dual{\ell}_i\sigma_i=\ell\sigma$.
\end{property}

%example 3: shows requirement for contraction check

 \begin{example}\label{ex:rootpair} %The following shows why the above is not necessarily  enough (we must check the original sources of the aux formulas, and see if those can be contracted), otherwise we might not save anything.
Although simultaneous unification of literals is necessary in order to achieve some compression, it may not be enough. In the last example, the literals were checked as they appeared when they were to be resolved against a unit $u$ which was to be postponed. However, it may be the case that they appeared this way (had a particular set of variables) because a series of unifiers $\sigma_1,\ldots,\sigma_n$ were applied to their original form $\ell'$ so that $\ell=\ell'\sigma_1\dots\sigma_n$. For example,

\begin{footnotesize}
\begin{prooftree}
\def\e{\mbox{\ $\vdash$\ }}
\AxiomC{$\eta_1$: $\dual{\ell}',t_1,t_2$ \e}
\AxiomC{$\eta_2$: \e $\dual{t_2}$}
\RightLabel{$\sigma_1$}
\BinaryInfC{$\eta_3$:  $(\dual{\ell}'\sigma_1),(t_1\sigma_1)$ \e}
\AxiomC{$\eta_4$: \e$\dual{t_1}$}
\RightLabel{$\sigma_2$}
\BinaryInfC{$\eta_5$:  $\dual{\ell}=((\dual{\ell}'\sigma_1)\sigma_2)$ \e}
\AxiomC{$\eta_6$: \e $\ell$}
\RightLabel{$\sigma_3$}
\BinaryInfC{$\psi$: $\bot$}
\end{prooftree}
\end{footnotesize}

Even if $\ell$ satisfied Property \ref{prop:pair}, the unifiers $\sigma_1$ and $\sigma_2$ might not be applied in the case of postponing resolution with $u$ (or more generally, because another unit clause has been postponed). The following proof provides a concrete example, where resolution with $u=\eta_2$ is to be postponed.
\begin{footnotesize}
\begin{prooftree}
\def\e{\mbox{\ $\vdash$\ }}
\AxiomC{$\eta_1$: $r(Y),p(X ~q(Y~b)), p(X~Y)$\e}
\AxiomC{$\eta_2$: \e $p(U~V)$}
\BinaryInfC{$\eta_3$: $r(V),p(U ~q(V~b))$\e}
\AxiomC{$\eta_4$: \e $r(W)$}
\BinaryInfC{$\eta_5$: $p(U ~q(W~b))$\e}
\AxiomC{$\eta_2$}
\BinaryInfC{$\psi$: $\bot$}
\end{prooftree}
\end{footnotesize}

Note that the literals resolved against $u$ are $p(X~Y)$ (in $\eta_1$) and $p(U ~q(V~b))$ (in $\eta_3$) and further that the latter is $\sigma_1 = \{X\setminus U, Y\setminus V\}$  applied to $p(X ~q(Y~b))$. These two formulas are unifiable via the substitution $\sigma_2 = \{X\setminus U, Y\setminus q(V~b) \}$. However, if resolution with $u$ is postponed, we will not apply the unification $\sigma_1$, and the original sources of these two formulas, $p(X~Y)$ and $p(X ~q(Y~b))$ can not be contracted. Thus we require roots of resolved formulas to be simultaneously unifiable, and we define this property formally below.
 \end{example}

\begin{property}
\label{prop:rootpair}
%Descendants must be pair-wise unifiable... TODO: this? Or leave this out and talk about it section 5?
Let $u$ be a unit clause with literal $\ell$, let $\dual{\ell}_1\in\Gamma_{v_1},\ldots,\dual{\ell}_n\in\Gamma_{v_n}$ be the literals that are unified with $\ell$ during resolution between $v_i$ and $u$ in a proof $\psi$. Let $\dual{\ell}_1^r$ be the original source of the literal $\dual{\ell}_1$: $\dual{\ell}_1^r\in v_{1'}$ such that there is a maximal sequence of unifications $s$ applied to $v_{1'}$ and its children in the proof so that eventually $\dual{\ell}_1=\dual{\ell}_1^r\sigma_1\ldots\sigma_s$.
%Then for every $\dual{\ell}_i^r$ and  $\dual{\ell}_j^r$ for $i\neq j$, there should be a unifier $\sigma_{i,j}$ such that after applying $\sigma_{i,j}$, $\dual{\ell}_i^r$ and $\dual{\ell}_j^r$ are syntactically equal.
%Then for every $\dual{\ell}_i^r$ and  $\dual{\ell}_j^r$ for $i\neq j$, there should be a unifier $\sigma_{i,j}$ such that after applying $\sigma_{i,j}$, $FV(\dual{\ell}_i^r\sigma_{i,j})=FV(\dual{\ell}_j^r)$ or $FV(\dual{\ell}_i^r)=FV(\dual{\ell}_j^r\sigma_{i,j})$.
Then there should be a unifier $\sigma$ such that for every $\dual{\ell}_i^r$, there exists a unifier $\sigma_{i}$ such that $\dual{\ell}_i^r\sigma_i=\ell\sigma$.
\end{property}

%example 4: requires FOSubstitution, introduces this concept
\begin{example}\label{ex:ambig} %Consider the following, which shows the dangers of ambiguous resolution in the first order case:\\
Lastly, we note that care must be taken when lowering a valid unit in order to ensure that the proof can still be completed. In particular, postponing resolution of a node $v$ with a unit $u$ may result in an \emph{ambiguous resolution}. A resolution is said to be ambiguous if there are more than one pair of literals $(\ell, \dual{\ell})$ with $\ell \in \Gamma_{\psi_l}$ and $\dual{\ell} \in \Gamma_{\psi_r}$ such that $(\ell, \dual{\ell})$ are unifiable. %and the conclusion is not known?
In the case of ambiguous resolution, any compression algorithm must take care to pick the appropriate pair  $(\ell, \dual{\ell})$ since either $\ell$ or $\dual{\ell}$ might share variables with other literals in the premises, and attempting to resolve away the wrong pair may ground (or otherwise modify) another literal $\ell'$ in such a way that $\ell'$ can no longer be unified with any other literal in the proof, resulting in the inability to complete the proof. Consider the following proof of $\psi$, where resolution with $u=\eta_2$ is to be postponed:
\begin{footnotesize}
\begin{prooftree}
\def\e{\mbox{\ $\vdash$\ }}
\def\defaultHypSeparation{\hskip .05in}

\AxiomC{$\eta_{2}$}
\AxiomC{$\eta_6$: \e$r(W~V)$}
\AxiomC{$\eta_4$: \e$r(X~c)$}
\AxiomC{$\eta_1$: $p(U),r(U~V),r(V~U),q(V)$\e}
\AxiomC{$\eta_2$: \e$p(c)$}
\BinaryInfC{$\eta_3$: $r(c~V),r(V~c),q(V)$\e}
%\AxiomC{$\eta_4$: \e$r(X~c)$}
\BinaryInfC{$\eta_5$: $r(c~X),q(X)$\e}
%\AxiomC{$\eta_6$: \e$r(W~V)$}
\BinaryInfC{$\eta_7$: $q(V)$\e}
\AxiomC{$\eta_8$: $p(Z)$\e$q(d)$}
\insertBetweenHyps{\hskip -1.5in}
\BinaryInfC{$\eta_9$: $p(Z)$\e}
\insertBetweenHyps{\hskip -0.8in}
\BinaryInfC{$\psi$: $\bot$}
\end{prooftree}
\end{footnotesize}

If we postpone $\eta_2$: $\vdash p(c)$, the first resolution in the compressed proof would be between the following two clauses:
\begin{equation}
p(U),r(U~V),r(V~U),q(V)\vdash 
\end{equation}
\begin{equation}
\vdash r(X~c) 
\end{equation}
Both $r(U~V)$ and $r(V~U)$ are unifiable with the literal in (2), and so this resolution is ambiguous. If we used $(r(U~V),r(X~c))$, then we would use the unifier $\sigma =\{U\setminus X, V\setminus c\}$, which would result in the following resolvent clause:
\begin{equation}
p(X),r(V~X),q(c)\vdash
\end{equation}
However, the original proof does not have a clause which contains $q(c)$ in the succedent, so it would be impossible to complete the proof. On the other hand, if we chose $(r(V~U),r(X~c))$, we would unify with $\sigma = \{V\setminus X, U\setminus c\}$, with which we could complete the proof:

\begin{footnotesize}
\begin{prooftree}
\def\defaultHypSeparation{\hskip .05in}
\def\e{\mbox{\ $\vdash$\ }}
\AxiomC{$\eta_9'$: \e$p(c)$}
\AxiomC{$\eta_6'$: $p(Z)$\e$q(d)$}

\AxiomC{$\eta_1'$: $p(U),r(U~V),r(V~U),q(V)$\e}
\AxiomC{$\eta_2'$: \e$r(X~c)$}
\BinaryInfC{$\eta_3'$: $p(c),r(c~X),q(X)$\e}
\AxiomC{$\eta_4'$: \e$r(W~V)$}
\BinaryInfC{$\eta_5'$: $p(c),q(V)$\e}
\insertBetweenHyps{\hskip -0.5in}
\BinaryInfC{$\eta_7'$: $p(c),p(Z)$\e}
\UnaryInfC{$\eta_8'$: $p(c)$\e}
\insertBetweenHyps{\hskip -0.8in}
\BinaryInfC{$\psi$: $\bot$}
\end{prooftree}
\end{footnotesize}
Note that this issue is not present in the propositional case since there is no unification, and resolution cannot affect any terms in a premise that are not removed during resolution. In practice, this issue can be avoided by careful book keeping which would not be necessary in the propositional case. %Several techniques are available to ensure the correct choice of ambiguous resolutions are always made, the simplest of which is described in Section \ref{sec:SimpleFOLU}.
%A method to avoid this issue is discussed in section \ref{sec:SimpleFOLU}
 \end{example}
