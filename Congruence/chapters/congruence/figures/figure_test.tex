\documentclass[a4paper,11pt,twoside]{memoir}
\usepackage{amsthm}
\usepackage{enumitem,amsmath,amssymb}

\usepackage{url}
\usepackage{hyperref}					% links in pdf
\usepackage{graphicx}            			% Figures
\usepackage{verbatim}            			% Code-Environment
\usepackage[linesnumbered,algochapter,noend,ruled]{algorithm2e} % Algorithm-Environment

\usepackage{pgf}					
\usepackage{tikz}					% tikz graphics
\usetikzlibrary{arrows,automata,positioning}
\usetikzlibrary{fit}
\usepackage{ngerman}
\usepackage[ngerman]{babel}
\usepackage{bibgerm,cite}       % Deutsche Bezeichnungen, Automatisches Zusammenfassen von Literaturstellen
\usepackage[ngerman]{varioref}  % Querverweise
% to use the german charset include cp850 for MS-DOS, ansinew for Windows and latin1 for Linux.
% \usepackage[latin1]{inputenc}

\pagenumbering{gobble}

\usepackage{../../../drawproof}

\begin{document}

%\input{compressproof_new_poster}
%$$
%\begin{array}{l l}
	%\bullet \text{ reflexivity:} & t = t \\
	%\bullet\text{ symmetry:} & t = s \rightarrow s = t \\
	%\bullet\text{ transitivity:} & t_1 = t_2 \wedge \ldots \wedge t_{n-1} = t_n \rightarrow t_1 = t_n \\
	%\bullet\text{ compatibility:} & \bigwedge_{i=1}^{n} t_i = s_i \rightarrow f(t_1,\ldots,t_n) = f(s_1,\ldots, s_n)
%\end{array}
%$$
\end{document}
