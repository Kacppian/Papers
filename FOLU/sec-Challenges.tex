
\section{First-Order Challenges}\label{sec:Challenges}


In this section, we describe challenges that have to be overcome in order to successfully adapt {\LowerUnits} to the first-order case. The first example illustrates the need to take unification into account. The other three examples discuss complex issues that can arise when unification is take into account in a naive way.


 \begin{example}\label{ex:unif} 
 Consider the following proof $\psi$, and note that the unit subproof $\eta_2$ is used twice. It is resolved once with $\eta_1$ (against the literal $p(W)$ and producing the child $\eta_3$) and once with $\eta_5$ (against the literal $p(X)$ and producing the root $\psi$).

\begin{footnotesize}
\begin{prooftree}
\def\e{\mbox{\ $\vdash$\ }}
\AxiomC{$\eta_1$: $p(W)$\e$q(Z)$}
\AxiomC{$\eta_2$: \e$p(Y)$}
\BinaryInfC{$\eta_3$: \e$q(Z)$}
\AxiomC{$\eta_4$: $p(X),q(Z)$\e}
\BinaryInfC{$\eta_5$: $p(X)$\e}
\AxiomC{$\eta_2$}
\BinaryInfC{$\psi$: $\bot$}
\end{prooftree}
\end{footnotesize}

\noindent
The result of deleting $\eta_2$ from $\psi$ is the proof $\dn{\psi}{\eta_2}$ shown below:

\begin{footnotesize}
\begin{prooftree}
\def\e{\mbox{\ $\vdash$\ }}
\AxiomC{$\eta'_1$: $p(W)$\e$q(Z)$}
\AxiomC{$\eta'_4$: $p(X),q(Z)$\e}
\BinaryInfC{$\eta'_5$ ($\psi'$): $p(W), p(X)$\e}
\end{prooftree}
\end{footnotesize}

\noindent
Unlike in the propositional case, where the literals that had been resolved against the unit are all syntactically equal, in the first-order case, this is not necessarily the case. As illustrated above, $p(W)$ and $p(X)$ are not syntactically equal. Nevertheless, they are unifiable. Therefore, in order to reintroduce $\eta'_2$, we may first perform a contraction, as shown below:
\begin{footnotesize}
\begin{prooftree}
\def\e{\mbox{\ $\vdash$\ }}
\AxiomC{$\eta_1'$: $p(W)$\e$q(Z)$}
\AxiomC{$\eta_4'$: $p(X),q(Z)$\e}
\BinaryInfC{$\eta_5'$: $p(X),p(Y)$\e}
\UnaryInfC{$\con{\eta_5'}{}{}$: $p(U)$\e}
\AxiomC{$\eta_2'$: \e$p(Y)$}
\BinaryInfC{$\psi^{\star}$: $\bot$}
\end{prooftree}
\end{footnotesize}
 \end{example}


 \begin{example}\label{ex:pairwise}

There are cases, as shown below, when the literals that had been resolved away are not unifiable, and then a contraction is not possible.

\begin{footnotesize}
\begin{prooftree}
\def\e{\mbox{\ $\vdash$\ }}
\AxiomC{$\eta_2$}
\AxiomC{$\eta_4$: $r(X),p(b)$\e $s(Y)$}
\AxiomC{$\eta_1$: $p(a)$\e$q(Y),r(Z)$}
\AxiomC{$\eta_2$: \e $p(X)$}
\BinaryInfC{$\eta_3$: \e$q(Y),r(Z)$}
\BinaryInfC{$\eta_5$: $p(b)$\e $s(Y),q(Y)$}
\AxiomC{$\eta_6$: $s(Y), q(Y)$\e}
\insertBetweenHyps{\hskip -0.5in}
\BinaryInfC{$\eta_7$: $p(b)$\e}
\insertBetweenHyps{\hskip -0.8in}
\BinaryInfC{$\psi$: $\bot$}
\end{prooftree}
\end{footnotesize}

\noindent
If we attempted to postpone the resolution inferences involving the unit $\eta_2$ (i.e. by deleting $\eta_2$ and reintroducing it with a single resolution inference in the bottom of the proof), a contraction of the literals $p(a)$ and $p(b)$ would be needed. Since these literals are not unifiable, the contraction is not possible. Note that, in principle, we could still lower $\eta_2$ if we resolved it not only once but twice when reintroducing it in the bottom of the proof. However, this would lead to no compression of the proof's length.
 \end{example}

\noindent
The observations above lead to the idea of requiring units to satisfy the following property before collecting them to be lowered.

\begin{definition}
\label{prop:pair}
Let $\eta$ be a unit with literal $\ell$ and let $\eta_1$, \ldots, $\eta_n$ be subproofs that are resolved with $\eta$ in a proof $\psi$, respectively, with resolved literals $\ell_1$, \ldots, $\ell_n$. 
$\eta$ is said to satisfy the \emph{pre-deletion unifiability property} in $\psi$ if $\ell_1$,\ldots,$\ell_n$, and $\dual{\ell}$ are unifiable.
\end{definition}



 \begin{example}\label{ex:rootpair}
Satisfaction of the pre-deletion unifiability property is not enough. Deletion of the units from a proof $\psi$ may actually change the literals that had been resolved away by the units, because fewer substitutions are applied to them. This is exemplified below:

\begin{footnotesize}
\begin{prooftree}
\def\e{\mbox{\ $\vdash$\ }}
\AxiomC{$\eta_1$: $r(Y),p(X, q(Y, b)), p(X, Y)$\e}
\AxiomC{$\eta_2$: \e $p(U, V)$}
\BinaryInfC{$\eta_3$: $r(V),p(U, q(V, b))$\e}
\AxiomC{$\eta_4$: \e $r(W)$}
\BinaryInfC{$\eta_5$: $p(U, q(W, b))$\e}
\AxiomC{$\eta_2$}
\BinaryInfC{$\psi$: $\bot$}
\end{prooftree}
\end{footnotesize}

\noindent
If $\eta$ is collected for lowering and deleted from $\psi$, we obtain the proof $\dn{\psi}{\eta}$:

\begin{footnotesize}
\begin{prooftree}
\def\e{\mbox{\ $\vdash$\ }}
\AxiomC{$\eta'_1$: $r(Y),p(X, q(Y, b)), p(X, Y)$\e}
\AxiomC{$\eta'_4$: \e $r(W)$}
\BinaryInfC{$\eta'_5 (\psi')$: $p(X, q(W, b)), p(X, W)$\e}
\end{prooftree}
\end{footnotesize}

\noindent
Note that, even though $\eta_2$ satisfies the pre-deletion unifiability property (since $p(X, q(Y, b))$ and $p(U, q(W, b))$ are unifiable), $\eta_2$ still cannot be lowered and reintroduced by a single resolution inference, because the corresponding modified post-deletion literals $p(X, q(W, b))$ and $p(X, W)$ are actually not unifiable.
\end{example}

The observation above leads to the following stronger property:

\begin{definition}
\label{prop:rootpair}
Let $\eta$ be a unit with literal $\ell_{\eta}$ and let $\eta_1$, \ldots, $\eta_n$ be subproofs that are resolved with $\eta$ in a proof $\psi$, respectively, with resolved literals $\ell_1$, \ldots, $\ell_m$. 
$\eta$ is said to satisfy the \emph{post-deletion unifiability property} in $\psi$ if $\ell_1^{\dagger\downarrow}$,\ldots,$\ell_m^{\dagger\downarrow}$, and $\dual{\ell_{\eta}^{\dagger}}$ are unifiable, where $\ell^{\dagger}$ is the literal in $\dn{\psi}{\eta}$ corresponding to $\ell$ in $\psi$ and $\ell_k^{\dagger\downarrow}$ is the descendant of $\ell_k^{\dagger}$ in the root of $\dn{\psi}{\eta}$.
\end{definition}

% \begin{definition}
% Let $\eta$ be a unit with literal $\ell_{\eta}$ and let $\eta_1$, \ldots, $\eta_n$ be subproofs that are resolved with $\eta$ in a proof $\psi$, respectively, with resolved literals $\ell_1$, \ldots, $\ell_m$. 
% $\eta$ is said to satisfy the \emph{ancestor unifiability property} in $\psi$ if $\ell_1^{\uparrow}$,\ldots,$\ell_m^{\uparrow}$, and $\dual{\ell_{\eta}}$ are unifiable, where $\ell^{\uparrow}$ is the highest ancestor of $\ell$ in $\psi$.
% \end{definition}


\begin{example}\label{ex:ambig} %Consider the following, which shows the dangers of ambiguous resolution in the first order case:\\
Lastly, we note that care must be taken when lowering a valid unit in order to ensure that the proof can still be completed. In particular, postponing resolution of a node $v$ with a unit $u$ may result in an \emph{ambiguous resolution}. A resolution is said to be ambiguous if there are more than one pair of literals $(\ell, \dual{\ell})$ with $\ell \in \Gamma_{\psi_l}$ and $\dual{\ell} \in \Gamma_{\psi_r}$ such that $(\ell, \dual{\ell})$ are unifiable. %and the conclusion is not known?
In the case of ambiguous resolution, any compression algorithm must take care to pick the appropriate pair  $(\ell, \dual{\ell})$ since either $\ell$ or $\dual{\ell}$ might share variables with other literals in the premises, and attempting to resolve away the wrong pair may ground (or otherwise modify) another literal $\ell'$ in such a way that $\ell'$ can no longer be unified with any other literal in the proof, resulting in the inability to complete the proof. Consider the following proof of $\psi$, where resolution with $u=\eta_2$ is to be postponed:
\begin{footnotesize}
\begin{prooftree}
\def\e{\mbox{\ $\vdash$\ }}
\def\defaultHypSeparation{\hskip .05in}

\AxiomC{$\eta_{2}$}
\AxiomC{$\eta_6$: \e$r(W~V)$}
\AxiomC{$\eta_4$: \e$r(X~c)$}
\AxiomC{$\eta_1$: $p(U),r(U~V),r(V~U),q(V)$\e}
\AxiomC{$\eta_2$: \e$p(c)$}
\BinaryInfC{$\eta_3$: $r(c~V),r(V~c),q(V)$\e}
%\AxiomC{$\eta_4$: \e$r(X~c)$}
\BinaryInfC{$\eta_5$: $r(c~X),q(X)$\e}
%\AxiomC{$\eta_6$: \e$r(W~V)$}
\BinaryInfC{$\eta_7$: $q(V)$\e}
\AxiomC{$\eta_8$: $p(Z)$\e$q(d)$}
\insertBetweenHyps{\hskip -1.5in}
\BinaryInfC{$\eta_9$: $p(Z)$\e}
\insertBetweenHyps{\hskip -0.8in}
\BinaryInfC{$\psi$: $\bot$}
\end{prooftree}
\end{footnotesize}

If we postpone $\eta_2$: $\vdash p(c)$, the first resolution in the compressed proof would be between the following two clauses:
\begin{equation}
\eta_1:~p(U),r(U~V),r(V~U),q(V)\vdash 
\end{equation}
\begin{equation}
\eta_4:~\vdash r(X~c) 
\end{equation}
Both $r(U~V)$ and $r(V~U)$ are unifiable with the literal in (2), and so this resolution is ambiguous. If we used $(r(U~V),r(X~c))$, then we would use the unifier $\sigma =\{U\setminus X, V\setminus c\}$, which would result in the following resolvent clause:
\begin{equation}
p(X),r(V~X),q(c)\vdash
\end{equation}
However, the original proof does not have a clause which contains $q(c)$ in the succedent, so it would be impossible to complete the proof. On the other hand, if we chose $(r(V~U),r(X~c))$, we would unify with $\sigma = \{V\setminus X, U\setminus c\}$, with which we could complete the proof:

\begin{footnotesize}
\begin{prooftree}
\def\defaultHypSeparation{\hskip .05in}
\def\e{\mbox{\ $\vdash$\ }}
\AxiomC{$\eta_9'$: \e$p(c)$}
\AxiomC{$\eta_6'$: $p(Z)$\e$q(d)$}

\AxiomC{$\eta_1'$: $p(U),r(U~V),r(V~U),q(V)$\e}
\AxiomC{$\eta_2'$: \e$r(X~c)$}
\BinaryInfC{$\eta_3'$: $p(c),r(c~X),q(X)$\e}
\AxiomC{$\eta_4'$: \e$r(W~V)$}
\BinaryInfC{$\eta_5'$: $p(c),q(V)$\e}
\insertBetweenHyps{\hskip -0.5in}
\BinaryInfC{$\eta_7'$: $p(c),p(Z)$\e}
\UnaryInfC{$\eta_8'$: $p(c)$\e}
\insertBetweenHyps{\hskip -0.8in}
\BinaryInfC{$\psi$: $\bot$}
\end{prooftree}
\end{footnotesize}
Note that this issue is not present in the propositional case since there is no unification, and resolution cannot affect any terms in a premise that are not removed during resolution. In practice, this issue can be avoided by careful book keeping which would not be necessary in the propositional case. %Several techniques are available to ensure the correct choice of ambiguous resolutions are always made, the simplest of which is described in Section \ref{sec:SimpleFOLU}.
%A method to avoid this issue is discussed in section \ref{sec:SimpleFOLU}
 \end{example}
