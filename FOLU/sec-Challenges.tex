


\section{First-Order Challenges}\label{sec:Challenges}

TODO by Jan (just writing some ideas so far--not yet final by any means)\\
{\bf todo: define the notion of set of literals resolved against a unit? might make things clearer}

In this section, we discuss the challenges introduced by adapting {\LowerUnits} to the first-order case. The first example illustrates how to extend {\LowerUnits} to first-order logic in the obvious way. Examples \ref{ex:pairwise} and on illustrate concerns that are introduced by the unification process that must be over come in order to successfully postpone resolution with a unit clause as a result of this extension.


%example 1: shows requirement for pair-wise unifiability with unit
 \begin{example} Resolution with a particular unit clause $u$, with literal $\ell$ may be performed with a clause $v$ provided $v$ contains $\dual{\ell}$ and there is some unifier $\sigma$ such that $\ell\sigma$ contains the same variables as $\dual{\ell}$ (or such that $\dual{\ell}\sigma$ contains the same variables as $\ell$). After applying $\sigma$ to the resolvents, the literals match syntactically, and so this behaves like the propositional case. Thus, the notion of looking for unifiable formulas to postpone resolution with $u$ is natural. Consider the following proof of $\psi$, where we will postpone resolution with $\eta_2$:

\begin{footnotesize}
\begin{prooftree}
\def\e{\mbox{\ $\vdash$\ }}
\AxiomC{$\eta_1$: $p(Y)$\e$q(Z)$}
\AxiomC{$\eta_2$: \e$p(Y)$}
\BinaryInfC{$\eta_3$: \e$q(Z)$}
\AxiomC{$\eta_4$: $p(X),q(Z)$\e}
\BinaryInfC{$\eta_5$: $p(X)$\e}
\AxiomC{$\eta_2$}
\BinaryInfC{$\psi$: $\bot$}
\end{prooftree}
\end{footnotesize}

In order to postpone resolution with $\eta_2$, the formulas $p(Y)$ (in $\eta_1$) and $p(X)$ (in $\eta_4$) will both remain after unifying these two nodes together. Unlike in the propositional case, where we could drop the repeated formula, in order to compress the proof soundly, we must first contract these formulas ($\eta_3'$) into a single literal ($\eta_4'$). Then we can finish the proof by reintroducing our postponed resolution, as in below.

\begin{footnotesize}
\begin{prooftree}
\def\e{\mbox{\ $\vdash$\ }}
\AxiomC{$\eta_1'$: $p(Y)$\e$q(Z)$}
\AxiomC{$\eta_2'$: $p(X),q(Z)$\e}
\BinaryInfC{$\eta_3'$: $p(X),p(Y)$\e}
\UnaryInfC{$\eta_4'$: $p(X)$\e}
\AxiomC{$\eta_5'$: \e$p(Y)$}
\BinaryInfC{$\psi$: $\bot$}
\end{prooftree}
\end{footnotesize}
 \end{example}


%example 2: shows requirement for pair-wise unifiability within all aux formulas

 \begin{example}\label{ex:pairwise} %The following example shows why we must check pair-wise unifiability with the literals resolved against the unit we're trying to lower. 
When attempting to lower a unit clause in the first order case, additional properties must be satisfied. In addition to the requirement that there is a unifier $\sigma$ that makes the unit clause syntactically equal to the formula it resolves away, we require that all such  unifiers between $u$ behave similarly in some sense. Consider the example below, where we consider postponing $\eta_2$ in the proof of $\psi$. 

\begin{tiny}
\begin{prooftree}
\def\e{\mbox{\ $\vdash$\ }}
\AxiomC{$\eta_2$}
\AxiomC{$\eta_1$: $p(a)$\e$q(Y),r(Z)$}
\AxiomC{$\eta_2$: \e $p(X)$}
\BinaryInfC{$\eta_3$: \e$q(Y),r(Z)$}
\AxiomC{$\eta_4$: $r(X),p(b)$\e $s(Y)$}
\BinaryInfC{$\eta_5$: $p(b)$\e $s(Y),q(Y)$}
\AxiomC{$\eta_6$: $s(Y), q(Y)$\e}
\BinaryInfC{$\eta_7$: $p(b)$\e}
\BinaryInfC{$\psi$: $\bot$}
\end{prooftree}
\end{tiny}

The literals resolved with $u = \eta_2$ are $p(a)$ (in $\eta_1$) and $p(b)$ (in $\eta_7$). If we attempt to postpone resolution, at the contraction step the clause would be $p(a),p(b)$. This clause cannot be contracted, as there is no unifier between these terms that would make their variable sets equal (in fact, there are no variables at all). Thus, it would be a waste of time to attempt to postpone resolution with $u$, and so we require any unit we wish to lower to satisfy the following property.
 \end{example}

\begin{property}
Things must be pair-wise unifiable... TODO: this? Or leave this out and talk about it section 5?
\end{property}

%example 3: shows requirement for contraction check

 \begin{example} %The following shows why the above is not necessarily  enough (we must check the original sources of the aux formulas, and see if those can be contracted), otherwise we might not save anything.
Although pair-wise unifiability of literals is necessary in order to achieve some compression, it is not enough. In the last example, the literals were checked as they appeared when they were to be resolved against a unit $u$ which was to be postponed. However, it may be the case that they appeared this way (had a particular set of variables) because of a series of unifiers $\sigma_1,\ldots,\sigma_n$ were applied to their original form $\ell'$ so that $\ell=\ell'\sigma_1\dots\sigma_n$ was pairwise unifiable (the last property was met) with all other literals, but these unifies $\sigma_i$ would not be applied in the case of postponing resolution with $u$ (or another unit clause). The following proof illustrates this case, where we would consider lowering $u=\eta_2$.
\begin{footnotesize}
\begin{prooftree}
\def\e{\mbox{\ $\vdash$\ }}
\AxiomC{$\eta_1$: $r(Y),p(X ~q(Y~b)), p(X~Y)$\e}
\AxiomC{$\eta_2$: \e $p(U~V)$}
\BinaryInfC{$\eta_3$: $r(V),p(U ~q(V~b))$\e}
\AxiomC{$\eta_4$: \e $r(W)$}
\BinaryInfC{$\eta_5$: $p(U ~q(W~b))$\e}
\AxiomC{$\eta_2$}
\BinaryInfC{$\psi$: $\bot$}
\end{prooftree}
\end{footnotesize}

Note that the formulas resolved against $u$ are $p(X~Y)$ (in $\eta_1$) and $p(U ~q(V~b))$ (in $\eta_3$). These two formulas are unifiable via the substitution $\{X\setminus U, Y\setminus q(V~b) \}$. However, if resolution with $u$ is postponed, we will not apply the unification $\{X\setminus U, Y\setminus V\}$ that is applied to $\eta_1$, and thus the descendants (the original sources) of these two formulas can no longer be unified and contracted. Thus we require descendants of resolved formulas to be pair-wise unifiable, and we call this property 2 below.
 \end{example}

\begin{property}
Descendants must be pair-wise unifiable... TODO: this? Or leave this out and talk about it section 5?
\end{property}

%example 4: requires FOSubstitution, introduces this concept
\begin{example}\label{ex:ambig} Consider the following, which shows the dangers of ambiguous resolution in the first order case:\\
\begin{tiny}
\begin{prooftree}
\def\e{\mbox{\ $\vdash$\ }}
\AxiomC{$\eta_1$: $p(U),r(U~V),r(V~U),q(V)$\e}
\AxiomC{$\eta_2$: \e$p(c)$}
\BinaryInfC{$\eta_3$: $r(c~V),r(V~c),q(V)$\e}
\AxiomC{$\eta_4$: \e$r(X~c)$}
\BinaryInfC{$\eta_5$: $r(c~X),q(X)$\e}
\AxiomC{$\eta_6$: \e$r(W~V)$}
\BinaryInfC{$\eta_7$: $q(V)$\e}
\AxiomC{$\eta_8$: $p(Z)$\e$q(d)$}
\BinaryInfC{$\eta_9$: $p(Z)$\e}
\AxiomC{$\eta_{2}$}
\BinaryInfC{$\psi$: $\bot$}
\end{prooftree}
\end{tiny}

Which if we lower $\eta_2$: $\vdash p(c)$, we would attempt to resolve  the following two sequents
$$p(U),r(U~V),r(V~U),q(V)\vdash$$
$$\vdash r(X~c)$$
Where we would have to be careful. If we used $r(U~V)$, then we would use the unifier $U->X, V->C$, which would result in the following:
$$p(X),r(V~X),q(c)\vdash$$
but the original proof does not have a method for resolving away $q(c)$, so we would not be able to complete the proof. On the other hand, if we chose $r(V~U)$, we would unify with $V->X, U->c$, with which we could complete the proof:

\begin{tiny}
\begin{prooftree}
\def\e{\mbox{\ $\vdash$\ }}
\AxiomC{$\eta_1'$: $p(U),r(U~V),r(V~U),q(V)$\e}
\AxiomC{$\eta_2'$: \e$r(X~c)$}
\BinaryInfC{$\eta_3'$: $p(c),r(c~X),q(X)$\e}
\AxiomC{$\eta_4'$: \e$r(W~V)$}
\BinaryInfC{$\eta_5'$: $p(c),q(V)$\e}
\AxiomC{$\eta_6'$: $p(Z)$\e$q(d)$}
\BinaryInfC{$\eta_7'$: $p(c),p(Z)$\e}
\UnaryInfC{$\eta_8'$: $p(c)$\e}
\AxiomC{$\eta_9'$: \e$p(c)$}
\BinaryInfC{$\psi$: $\bot$}
\end{prooftree}
\end{tiny}

A method to avoid this issue is discussed in section \ref{sec:SimpleFOLU}
 \end{example}
