\documentclass{llncs}

\usepackage{xcolor}
\usepackage{enumitem,amsmath,amssymb}
\usepackage{breakurl}    % used for \url and \burl
\usepackage[linesnumbered,boxed,noline,noend]{algorithm2e}
\def\defaultHypSeparation{\hskip.1in}

\usepackage{tikz}
\usepackage{subfig}
\usepackage{array,booktabs,multirow}
\usepackage{placeins}

\usepackage{logictools}
\usepackage{prooftheory}
\usepackage{comment}
\usepackage{mathenvironments}
\usepackage{drawproof}
\usepackage{bussproofs}


\renewcommand{\topfraction}{0.85}
\renewcommand{\textfraction}{0.1}
\renewcommand{\floatpagefraction}{0.75}

\title{Towards the Compression of First-Order Resolution Proofs by Lowering Unit Clauses}

\author{
  Jan Gorzny\inst{1}
  \thanks{Supported by the Google Summer of Code 2014 program.}
  \and 
  Bruno Woltzenlogel Paleo\inst{2}
  \thanks{Supported by the Austrian Science Fund, project P24300.}
}

\authorrunning{J.\~Gorzny \and B.\~Woltzenlogel Paleo}

\institute{
  University of Victoria, Canada \\
  \email{jgorzny@uvic.ca}
  \and 
  Vienna University of Technology, Austria \\
  \email{bruno@logic.at}
}




\begin{document}

\maketitle


\begin{abstract}
The recently developed {\LowerUnits} algorithm compresses
propositional resolution proofs generated by SAT- and SMT-solvers by lowering (i.e. postponing) resolution inferences involving unit clauses (i.e. clauses having exactly one literal). This paper describes a generalization of this algorithm to the case of first-order resolution proofs generated by automated theorem provers. An empirical evaluation of a simplified version of this algorithm on hundreds of proofs shows promising results.
\end{abstract}


\setcounter{footnote}{0}

\section{Introduction}

Most of the effort in automated reasoning so far has been dedicated to the design and implementation of proof systems and efficient theorem proving procedures. As a result, saturation-based first-order automated theorem provers have achieved a high degree of maturity, with resolution \cite{Robinson} and superposition \cite{todo} being among the most common underlying proof calculi. Proof production is an essential feature of modern state-of-the-art provers and proofs are crucial for applications where the user requires certification of the answer provided by the prover. Nevertheless, efficient proof production is non-trivial \cite{SchultzAPPA}, and it is to be expected that the best, most efficient, provers do not necessarily generate the best, least redundant, proofs. And while the foundational problem of simplicity of proofs can be traced back at least to Hilbert's 24th Problem \cite{Hilbert}, the maturity of automated deduction has made it particularly relevant today. Therefore, it is a timely moment to develop methods that post-process and simplify proofs. 

For proofs generated by SAT- and SMT-solvers, which use propositional resolution as the basis for the DPLL and CDCL decision procedures, there is now a wide variety of proof compression techniques. Algebraic properties of the resolution
operation that might be useful for compression were investigated in \cite{bwp10}.
Compression algorithms based on rearranging and sharing chains of resolution inferences have been
developed in \cite{Amjad07} and \cite{Sinz}.  Cotton \cite{CottonSplit} proposed an algorithm that
compresses a refutation by repeteadly splitting it into a proof of a heuristically chosen literal $\ell$
and a proof of $\dual{\ell}$, and then resolving them to form a new refutation.  The {\ReduceReconstruct} algorithm \cite{RedRec} searches for locally redundant
subproofs that can be rewritten into subproofs of stronger clauses and with fewer resolution steps.
A linear time proof compression algorithm based on partial
regularization was proposed in \cite{RP08} and improved in \cite{LURPI}. Furthermore, \cite{LURPI} also described a new linear time algorithm called {\LowerUnits}, which delays resolution with unit clauses.

In contrast, for first-order theorem provers, there has been up to now (to the best of our knowledge) no attempt to design and implement an algorithm capable of taking a first-order resolution DAG-proof and efficiently simplifying it, outputting a possibly shorter pure first-order resolution DAG-proof. There are algorithms aimed at simplifying first-order sequent calculus tree-like proofs, based on cut-introduction \cite{BrunoLPAR,Hetzl}, and while in principle resolution DAG-proofs can be translated to sequent-calculus tree-like proofs (and then back), such translations lead to undesirable efficiency overheads. There is also an algorithm \cite{LPARCzech} that looks for terms that occur often in any TSTP \cite{TPTP} proof (including first-order resolution DAG-proofs) and introduces abbreviations for these terms. However, as the definitions of the abbreviations are not part of the output proof, it cannot be checked by a pure first-order resolution proof checker.

In this paper, we initiate the process of lifting propositional proof compression techniques to the first-order case, starting with the simplest known algorithm: {\LowerUnits} (described in Section \ref{sec:PropositionalLU}). As shown in Section \ref{sec:Challenges}, even for this simple algorithm, the fact that first-order resolution makes use of unification leads to many challenges that simply do not exist in the propositional case. In Section \ref{sec:FOLU} we describe a sophisticated algorithm that overcomes these challenges. Furthermore, in Section \ref{sec:SimpleFOLU} we describe a simpler version of this algorithm, which is easier to implement and possibly more efficient, at the cost of compressing less. In Section \ref{sec:exp} we present experimental results obtained by applying the simpler algorithm on hundreds of proofs generated with the {\SPASS} theorem prover \cite{SPASS}. The next section introduces the first-order resolution calculus using notations that are more convenient for describing proof transformation operations.


\newcommand{\freevar}[1]{\mathrm{FV}(#1)}

\section{The Resolution Calculus}

We assume that there are infinitely many variable symbols (e.g. $X$, $Y$, $Z$, $X_1$, $X_2$, \ldots), constant symbols (e.g. $a$, $b$, $c$, $a_1$, $a_2$, \ldots), function symbols of every arity (e.g $f$, $g$, $f_1$, $f_2$, \ldots) and predicate symbols of every arity (e.g. $p$, $q$, $p_1$, $p_2$,\ldots). A \emph{term} is any variable, constant or the application of an $n$-ary function symbol to $n$ terms.
An \emph{atomic formula} (\emph{atom}) is the application of an $n$-ary predicate symbol to $n$ terms. A \emph{literal} is an atom or the negation of an atom. The
\emph{complement} of a literal $\ell$ is denoted $\dual{\ell}$ (i.e. for any atom $p$,
$\dual{p} = \neg p$ and $\dual{\neg p} = p$). The set of all literals is denoted $\mathcal{L}$. A
\emph{clause} is a multiset of literals. $\bot$ denotes the \emph{empty clause}.
$\freevar{t}$ (resp. $\freevar{\ell}$, $\freevar{\clause}$) denotes the set of variables in the term $t$ (resp. in the literal $\ell$ and in the clause $\clause$).
A \emph{substitution} $\{ X_1\backslash t_1, X_2 \backslash t_2, \ldots \}$ is a mapping from variables $\{ X_1, X_2, \ldots \}$ to, respectively, terms $\{t_1, t_2, \ldots \}$. The application of a substitution $\sigma$ to a term $t$, a literal $\ell$ or a clause $\clause$ results in, respectively, the term $t \sigma$, the literal $\ell \sigma$ or the clause $\clause \sigma$, obtained from $t$, $\ell$ and $\clause$ by replacing all occurrences of the variables in $\sigma$ by the corresponding terms in $\sigma$. The set of all substitutions is denoted $\mathcal{S}$. If  A \emph{unifier} of a set of literals is a substitution that makes all literals in the set equal.
A \emph{resolution proof} is a directed acyclic graph of clauses where the edges correspond to the inference rules of resolution and contraction (as explained in detail in Definition \ref{def:proof}). A \emph{resolution refutation} is a resolution proof with root $\bot$.


\newcommand{\axiom}[1]{\widehat{#1}}
\newcommand{\n}{v}
\newcommand{\raiz}[1]{\rho(#1)}

% Contraction
\newcommand{\con}[3]{\lfloor #1 \rfloor_{#2}^{#3}}

\begin{definition}[First-Order Resolution Proof] 
\label{def:proof} \hfill \\
A directed acyclic graph $\langle V, E, \clause \rangle$, where $V$ is a set of nodes and $E$ is a
set of edges labeled by literals and substitutions (i.e. $E \subset V \times \mathcal{L} \times \mathcal{S} \times V$ and $\n_1
\xrightarrow[\sigma]{\ell} \n_2$ denotes an edge from node $\n_1$ to node $\n_2$ labeled by the literal $\ell$ and the substitution $\sigma$), is a
proof of a clause $\clause$ iff it is inductively constructible according to the following cases:
%
\begin{itemize}
  \item \textbf{Axiom:} If $\Gamma$ is a clause, $\axiom{\Gamma}$ denotes some proof $\langle \{ \n \}, \varnothing,
    \Gamma \rangle$, where $\n$ is a new (axiom) node.
  \item \textbf{Resolution:} If $\psi_L$ is a proof $\langle V_L, E_L, \clause_L \rangle$ with $\ell_L \in \clause_L$ and
    $\psi_R$ is a proof $\langle V_R, E_R, \clause_R \rangle$ with $\ell_R \in \clause_R$, and 
    $\sigma_L$ and $\sigma_R$ are substitutions such that
    $\ell_L \sigma_L = \dual{\ell_R} \sigma_R$ and
    $\freevar{\left( \clause_L \setminus \left\{ \ell_L \right\} \right) \sigma_L} \cap 
     \freevar{\left( \clause_R
                    \setminus \left\{ \ell_R \right\} \right) \sigma_R} = \emptyset$, 
    then
    $\psi_L \odot_\ell \psi_R$ denotes a proof $\langle V, E, \Gamma \rangle$ s.t.
    \begin{align*}
      V &= V_L \cup V_R \cup \{\n \} \\
      E &= E_L \cup E_R \cup
                    \left\{ \raiz{\psi_L} \xrightarrow[\sigma_L]{\ell_L} \n, 
                            \raiz{\psi_R} \xrightarrow[\sigma_R]{\ell_R} \n \right\} \\
     \Gamma &= \left( \clause_L \setminus \left\{ \ell_L \right\} \right) \sigma_L \cup \left( \clause_R
                    \setminus \left\{ \ell_R \right\} \right) \sigma_R 
    \end{align*}
    where $\n$ is a new (resolution) node and $\raiz{\varphi}$ denotes the root node of $\varphi$.
  \item \textbf{Contraction:} If $\psi'$ is a proof $\langle V', E', \clause' \rangle$ and $\sigma$ is a unifier of $\{\ell_1, \ldots \ell_n\}$ with $\{\ell_1, \ldots \ell_n\} \subseteq \clause'$ and $\ell = \ell_k \sigma$ ($1 \leq k \leq n$), 
  then $\con{\psi}{\sigma}{\ell}$ denotes a proof $\langle V, E, \Gamma \rangle$ s.t.
    \begin{align*}
      V &= V' \cup \{\n \} \\
      E &= E' \cup \{ \raiz{\psi'} \xrightarrow[\sigma]{\ell} \n \} \\
     \Gamma &= (\clause' \setminus \{ \ell_1, \ldots \ell_n \} ) \sigma \cup \{ \ell \}
    \end{align*}
    where $\n$ is a new (contraction) node and $\raiz{\varphi}$ denotes the root node of $\varphi$.
  \qed
\end{itemize}
\end{definition}


\newcommand{\Vertices}[1]{V_{#1}}
\newcommand{\Edges}[1]{E_{#1}}
\newcommand{\Conclusion}[1]{\clause_{#1}}

\noindent
If $\psi = \varphi_L \odot_{\ell} \varphi_R$, then $\varphi_L$ and $\varphi_R$ are \emph{direct
subproofs} of $\psi$ and $\psi$ is a \emph{child} of both $\varphi_L$ and $\varphi_R$. The
transitive closure of the direct subproof relation is the \emph{subproof} relation. A subproof which
has no direct subproof is an \emph{axiom} of the proof.
%
$\Vertices{\psi}$, $\Edges{\psi}$ and $\Conclusion{\psi}$
denote, respectively, the nodes, edges and proved clause (conclusion) of $\psi$.


\section{The Propositional Lower Units}
\label{sec:PropositionalLU}


\SetKwFunction{Rec}{delete}
\SetKw{Let}{let}

\begin{algorithm}[bt]
  \KwIn{a proof $\varphi$}
  \KwIn{$D$ a set of subproofs}
  \KwOut{a proof $\varphi'$ obtained by deleting the subproofs in $D$ from $\varphi$}
  \BlankLine

  \newcommand{\fixL}{\ensuremath{\varphi'_L}}
  \newcommand{\fixR}{\ensuremath{\varphi'_R}}

  \uIf{$\varphi \in D$ or $\raiz{\varphi}$ has no premises}{
    \Return{$\varphi$} \;
  }
  \BlankLine

  \Else{
    \Let{$\varphi_L$, $\varphi_R$ and $\ell$} be such that
      $\varphi = \varphi_L \odot_\ell \varphi_R$ \;
    \Let{$\varphi'_L = $ \Rec{$\varphi_L$,$D$}} \;
    \Let{$\varphi'_R = $ \Rec{$\varphi_R$,$D$}} \;
    \BlankLine

    \uIf{$\varphi'_L \in D$}
      { \Return{\fixR} \; }
    \uElseIf{$\varphi'_R \in D$}
      { \Return{\fixL} \; }
    \BlankLine

    \uElseIf{$\dual{\ell} \notin \Conclusion{\fixL}$}
      { \Return{\fixL} \; }
    \uElseIf{$\ell \notin \Conclusion{\fixR}$}
      { \Return{\fixR} \; }
    \BlankLine

    \Else{ \Return{ \fixL~$\odot_\ell$~\fixR} \; }
  }

  \caption[.]{\FuncSty{delete}}
  \label{algo:del}
\end{algorithm}

The deletion algorithm is a minor variant of the \textsc{Reconstruct-Proof} algorithm presented in
\cite{RP11}.
The basic idea is to traverse the proof in a top-down manner, replacing
each subproof having one of its premises marked for deletion (i.e. in $D$) by its other direct
subproof. The special case when both $\varphi'_L$ and $\varphi'_R$ belong to $D$ is treated rather
implicitly and deserves an explanation: in such a case, one might intuitively expect the result
$\varphi'$ to be undefined and arbitrary. Furthermore, to any child of $\varphi$, $\varphi'$ ought
to be seen as if it were in $D$, as if the deletion of $\varphi'_L$ and $\varphi'_R$ propagated to
$\varphi'$ as well. Instead of assigning some arbitrary proof to $\varphi'$ and adding it to $D$,
the algorithm arbitrarily returns (in line 8) $\varphi'_R$ (which is already in $D$) as the result
$\varphi'$. In this way, the propagation of deletion is done automatically and implicitly. For
instance, the following hold:
\begin{align}
  \dn{\varphi_1 \odot_\ell \varphi_2}{\varphi_1, \varphi_2} &= \varphi_2 \label{eq:exampledel1} \\
\dn{\varphi_1 \odot_\ell \varphi_2 \odot_{\ell'} \varphi_3}{\varphi_1, \varphi_2} &=
  \dn{\varphi_3}{\varphi_1, \varphi_2} \label{eq:exampledel2}
\end{align}
A side-effect of this clever implicit propagation of deletion is that the actual result of deletion
is only meaningful if it is not in $D$. In the example (\ref{eq:exampledel1}), as $\dn{\varphi_1
\odot_\ell \varphi_2}{\varphi_1, \varphi_2} \in \{\varphi_1, \varphi_2\} $, the actual resulting
proof is meaningless. Only the information that it is a deleted subproof is relevant, as it suffices
to obtain meaningful results as shown in (\ref{eq:exampledel2}).

\begin{proposition} \label{prop:del_assoc}
For any proof $\psi$ and any sets $A$ and $B$ of $\psi$'s subproofs,
either $\dn{\psi}{A \cup B}  \in A \cup B$
and    $\dn{\dn{\psi}{A}}{B} \in A \cup B$,
or     $\dn{\psi}{A \cup B} = \dn{\dn{\psi}{A}}{B}$.
\end{proposition}




\newcommand{\pedge}[3]{\ensuremath{\raiz{#1} \xrightarrow{#2} \raiz{#3}}}



When a subproof $\varphi$ has more than one child in a proof $\psi$, it may be possible to \emph{factor} all
the corresponding resolutions: a new proof is
constructed by removing $\varphi$ from $\psi$ and reintroducing it later. The resulting proof is smaller because $\varphi$ participates in a single resolution inference in it (i.e. it has a single child), while in the original proof it participates in as many resolution inferences as the number of children it had. Such a factorization is called \emph{lowering} of $\varphi$, because its delayed reintroduction makes $\varphi$ appear at the bottom of the resulting proof. 

Formally, a subproof $\varphi$ in a proof $\psi$ can be lowered if there exists a proof
$\psi'$ and a literal $\ell$ such that $\psi' = \dn{\psi}{\varphi} \odot_{\ell} \varphi$ and
$\Conclusion{\psi'} \subseteq \Conclusion{\psi}$. It has been noted in \cite{LURPI} that $\varphi$ can always be lowered if it is a \emph{unit}: its conclusion clause has only one literal. This led to the invention of the {\LowerUnits} algorithm, which lowers every unit with more than one child, taking care to reintroduce units in
an order corresponding to the subproof relation: if a unit $\varphi_2$ is a subproof of a unit
$\varphi_1$ then $\varphi_2$ has to be reintroduced later than (i.e. below) $\varphi_1$.

A possible presentation of {\LowerUnits} is shown in Algorithm \ref{algo:LU}. Units are collected
during a first traversal. As this traversal is bottom-up, units are stored in a queue. The traversal
could have been top-down and units stored in a stack. Units are effectively deleted during a second,
top-down traversal. The last for-loop performs the reintroduction of units.

\begin{algorithm}[bt]
  \KwIn {a proof $\psi$}
  \KwOut{a compressed proof $\psi'$}
  \BlankLine

  \SetKwData{Units}{Units}
  \Units $\leftarrow \varnothing$ \;
  \BlankLine

  \For{every subproof $\varphi$ in a bottom-up traversal}{
    \If{$\varphi$ is a unit and has more than one child}{Enqueue $\varphi$ in \Units \; }
  }
  \BlankLine

  $\psi' \leftarrow $ \Rec{$\psi$,$\Units$} \;
  \BlankLine

  \For{every unit $\varphi$ in \Units}{
    \Let{$\{\ell\} = \Conclusion{\varphi}$} \;
    \lIf{$\dual{\ell} \in \Conclusion{\psi'}$}{
    $\psi' \leftarrow \psi' \odot_\ell \varphi$ \;}
  }

  \caption{\LowerUnits}
  \label{algo:LU}
\end{algorithm}



\section{First-Order Challenges}\label{sec:Challenges}

TODO by Jan (just writing some ideas so far--not yet final by any means)\\
{\bf Does this belong here?And is this what you had in mind for interesting examples? And are the proof formatted correctly, or should I change them? aside from the first one going over the margin right now of course}\\ 

In this section, we discuss additional requirements for lowering a unit formula in the first order case that are not required in the propositional case.

%example 1: shows requirement for pair-wise unifiability with unit
%obvious - skip?

%example 2: shows requirement for pair-wise unifiability within all aux formulas

 \begin{example} The following example shows why we must check pair-wise unifiability with the literals resolved against the unit we're trying to lower.

% \begin{tiny}
% \begin{prooftree}
% \def\e{\mbox{\ $\vdash$\ }}
% \AxiomC{$\eta_2$}
% \AxiomC{$\eta_1$: $p(a)$\e$q(Y),r(Z)$}
% \AxiomC{$\eta_2$: \e $p(X)$}
% \BinaryInfC{$\eta_3$: \e$q(Y),r(Z)$}
% \AxiomC{$\eta_4$: $r(X),p(b)$\e $s(Y)$}
% \BinaryInfC{$\eta_5$: $p(b)$\e $s(Y),q(Y)$}
% \AxiomC{$\eta_6$: $s(Y), q(Y)$\e}
% \BinaryInfC{$\eta_7$: $p(b)$\e}
% \BinaryInfC{$\psi$: $\bot$}
% \end{prooftree}
% \end{tiny}
 \end{example}


%example 3: shows requirement for contraction check

 \begin{example} The following shows why the above is not necessarily  enough (we must check the original sources of the aux formulas, and see if those can be contracted), otherwise we might not save anything.
% \begin{footnotesize}
% \begin{prooftree}
% \def\e{\mbox{\ $\vdash$\ }}
% \AxiomC{$\eta_1$: $r(Y),p(X ~q(Y~b)), p(X~Y)$\e}
% \AxiomC{$\eta_2$: \e $p(U~V)$}
% \BinaryInfC{$\eta_3$: $r(V),p(U ~q(V~b))$\e}
% \AxiomC{$\eta_4$: \e $r(W)$}
% \BinaryInfC{$\eta_5$: $p(U ~q(W~b))$\e}
% \AxiomC{$\eta_2$}
% \BinaryInfC{$\psi$: $\bot$}
% \end{prooftree}
% \end{footnotesize}
 \end{example}


%example 4: requires FOSubstitution, introduces this concept?



\section{First-Order LowerUnits} \label{sec:FOLU}

{\LowerUnits} does not lower every lowerable subproof. In particular, it does not take into
account the already lowered subproofs. For instance, if a unit $\varphi_1$ proving $\{a\}$ has
already been lowered, a subproof $\varphi_2$ with conclusion $\{\neg a, b\}$ may be lowered as well and
reintroduced above $\varphi_1$. The posterior reintroduction of $\varphi_1$ will resolve away $\neg a$ and guarantee that it does not occur in the resulting proof's conclusion. But care must also be taken not to lower $\varphi_2$ if $\neg a$ is a valent literal of
$\varphi_2$, otherwise $a$ will undesirably occur in the resulting proof's conclusion.

\begin{definition}[Univalent subproof]
A subproof $\varphi$ in a proof $\psi$ is \emph{univalent} w.r.t. a set $\Delta$ of literals iff
$\varphi$ has exactly one valent literal $\ell$ in $\psi$, $\ell \notin \Delta$ and
$\Conclusion{\varphi} \subseteq \Delta \cup \left\{ \ell \right\}$. $\ell$ is called the \emph{univalent
literal} of $\varphi$ in $\psi$ w.r.t.  $\Delta$.
\end{definition}

The principle of {\LowerUnivalents} is to lower all univalent subproofs. Having only one valent literal makes them behave essentially like units w.r.t. the technique of lowering. $\Delta$ is
initialized to the empty set. Then the complements of the univalent literals are incrementally added to
$\Delta$. Proposition \ref{prop:LUniv} ensures that the conclusion of the resulting proof
subsumes the conclusion of the original one.

\begin{proposition} \label{prop:LUniv}
Given a proof $\psi$, if 
%for an integer $n$
there is a sequence $U = (\varphi_1 \ldots \varphi_n)$
of $\psi$'s subproofs and a sequence $(\ell_1 \ldots \ell_n)$ of literals such that $\forall i \in
[1 \ldots n]$, $\ell_i$ is the univalent literal of $\varphi_i$ w.r.t. $\Delta_{i-1} =
\{\dual{\ell_1} \ldots \dual{\ell_{i-1}}\}$, then the conclusion of $$ \psi' = \dn{\psi}{U}
\odot_{\ell_n} \varphi_n \ldots \odot_{\ell_1} \varphi_1 $$ subsumes the conclusion of $\psi$.
\end{proposition}

\begin{proof}
The proposition is proven by induction on $n$, along with the fact that $\dn{\psi}{U} \notin U$.
For $n = 0$, $U = \varnothing$ and the properties trivially hold. Suppose a subproof
$\varphi_{n+1}$ of $\psi$ is univalent w.r.t. $\Delta_n$, with univalent literal $\ell_{n+1}$.
Because $\ell_{n+1} \notin \Delta_n$, there exists a subproof of $\dn{\psi}{U}$ with conclusion
containing $\dual{\ell_{n+1}}$, and therefore $\dn{\dn{\psi}{U}}{\varphi_{n+1}} \notin U \cup
\{\varphi_{n+1}\}$.  Let $\Gamma$ be the conclusion of $\dn{\psi}{U}$. The conclusion of $ \psi' =
\dn{\psi}{U \cup \{\varphi_{n+1}\}} = \dn{\dn{\psi}{U}}{\varphi_{n+1}} $ is included in $\Gamma \cup
\{\dual{\ell_{n+1}}\}$. The conclusion of $\psi' \odot_{\ell_{n+1}} \varphi_{n+1}$ is included in
$\Gamma \cup \Delta_n$. As $\Gamma \subseteq \Conclusion{\psi} \cup \Delta_n$, the conclusion of
$\psi' \odot_{\ell_{n+1}} \varphi_{n+1} \ldots \odot_{\ell_1} \varphi_1$ is included in
$\Conclusion{\psi}$. \qed
\end{proof}

For this principle to lead to proof compression, it is important to take care
of the mutual inclusion of univalent subproofs.
%not only of the order in which subproofs are collected for lowering but also of deleting all
%already collected univalent subproofs from the next subproof $\psi_i$ before reintroducing it.
Suppose, for instance, that $\varphi_i, \varphi_j, \varphi_k \in U$, $i < j < k$, $\varphi_j$ is a
subproof of $\varphi_i$ but not a subproof of $\dn{\psi}{\varphi_i}$, and $\dual{\ell_j} \in
\Conclusion{\varphi_k}$.  In this case, $\varphi_j$ will have one more child in
$$
\dn{\psi}{U} \odot_{\ell_n} \varphi_n \ldots \odot_{\ell_k} \varphi_k \ldots \odot_{\ell_j} \varphi_j \ldots \odot_{\ell_i} \varphi_i \ldots \odot_{\ell_1} \varphi_1
$$
than in the original proof $\psi$. The additional child is created when $\varphi_j$ is reintroduced.
All the other children are reintroduced with the reintroduction of $\varphi_i$, because
$\varphi_j$ was not deleted from $\varphi_i$.

To solve this issue, {\LowerUnivalents} traverses the proof in a top-down manner and simultaneously
deletes already collected univalent subproofs, as sketched in Algorithm \ref{algo:LUniv}.  


\SetKwData{Univ}{Univalents}
\begin{algorithm}[bt]
  \KwIn {a proof $\psi$}
  \KwOut{a compressed proof $\psi'$}
  \BlankLine

  \SetKw{Push}{push}
  \SetKw{Pop} {pop}

  \Univ $\leftarrow \varnothing$ \;
  $\Delta \leftarrow \varnothing$ \;
  \BlankLine

  \For{every subproof $\varphi$, in a top-down traversal \label{line:LUniv:step1begin} }{
    $\psi' \leftarrow$ \Rec{$\varphi$,\Univ} \label{line:LUniv:delete} \;
    \If{$\psi'$ is univalent w.r.t. $\Delta$ \label{line:LUniv:lunivtest} }{
      \Let{$\ell$} be the univalent literal \;
      \Push $\dual{\ell}$ onto $\Delta$ \label{line:LUniv:pushDelta} \;
      \Push $\psi'$     onto \Univ \label{line:LUniv:step1end} \;
    }
  }
  \BlankLine

  \tcp{At this point, $\psi' = \dn{\psi}{\Univ}$}
  \While{\Univ $\neq \varnothing$}{ \label{line:LUniv:reintroducebegin}
    $\varphi \leftarrow$ \Pop from \Univ \;
    $\ell \leftarrow$ \Pop from $\Delta$ \;
    \lIf{$\ell \in \Conclusion{\psi'}$ \label{line:LUniv:testreintroduce} }{
    $\psi' \leftarrow \varphi \odot_\ell \psi'$ \;}
  }

  \caption{Simplified \LowerUnivalents}
  \label{algo:LUniv}
\end{algorithm}


Figure \ref{fig:exluniv} shows an example proof and the result of compressing it with \LowerUnivalents. The top-down traversal starts with the leaves (axioms) and only visits a child when all its parents have already been visited. Assuming the unit with conclusion $\{a\}$ is the first visited leaf, it passes the univalent test in line \ref{line:LUniv:lunivtest}, is marked for lowering (line \ref{line:LUniv:step1end}) and the complement of its univalent literal is pushed onto $\Delta$ (line \ref{line:LUniv:pushDelta}). When the subproof with
conclusion $\{\dual{a},b\}$ is considered, $\Delta = \{\dual{a}\}$. As this subproof has only one
valent literal $b \notin \Delta$ and $\{\dual{a},b\} \subseteq \Delta \cup \{b\}$, it is
marked for lowering as well. At this point, $\Delta = \{\dual{a}, \dual{b}\}$, \texttt{Univalents} contains the two subproofs marked for lowering and $\psi'$ is the subproof with conclusion $\{\dual{a}, \dual{b}\}$ shown in Subfig. (b) (i.e. the result of deleting the two marked subproofs from the original proof in Subfig. (a)). No other subproof is univalent; no other subproof is marked for lowering. The final compressed proof (Subfig. (b)) is obtained by reintroducing the two univalent subproofs that had been marked (lines \ref{line:LUniv:reintroducebegin} -- \ref{line:LUniv:testreintroduce}). It has one resolution less than the original. This is so because the subproof with conclusion $\{\dual{a},b\}$ had been used (resolved) twice in the original proof, but lowering delays its use to a point where a single use is sufficient.

\begin{figure}[htb]
  \centering
  \subfloat[Original proof]{
    \centering
    \begin{tikzpicture}

      \rootnode;
      \withchildren{root} {r0}{\dual{a}}  {unit}{a};
      \withchildren{r0}   {r1}{\dual{a},c} {r2}{\dual{a},\dual{c}};
      \withchildren{r1}   {a0}{\dual{b},c} {low}{\dual{a},b};

      \proofnode[above right of=r2] {a1} {\dual{a},\dual{b},\dual{c}};
      \drawchildren {r2} {low} {a1};

    \end{tikzpicture}
  } \qquad
  \centering
  \subfloat[Compressed proof]{
    \centering
    \begin{tikzpicture}

      \rootnode;
      \withchildren{root} {r0}{\dual{a}}          {unit}{a};
      \withchildren{r0}   {r1}{\dual{a},\dual{b}} {low}{\dual{a},b};
      \withchildren{r1}   {a0}{\dual{b},c}        {a1}{\dual{a},\dual{b},\dual{c}};

    \end{tikzpicture}
  }
\caption{Example of proof crompression by \LowerUnivalents} 
\label{fig:exluniv}
\end{figure}


% Discussion of optimizations follow

Although the
call to \FuncSty{delete} inside the first loop (line \ref{line:LUniv:step1begin} to
\ref{line:LUniv:step1end}) suggests quadratic time complexity, this loop (line
\ref{line:LUniv:step1begin} to \ref{line:LUniv:step1end}) can be (and has been) actually implemented
as a recursive function extending a recursive implementation of \FuncSty{delete}. With such an
implementation, {\LowerUnivalents} has a time complexity linear w.r.t. the size of the proof, assuming the
univalent test (at line \ref{line:LUniv:lunivtest}) is performed in constant bounded time. 


Determining whether a literal is valent is expensive. But thanks to Proposition \ref{prop:valentactive},
subproofs with one active literal which is not in $\Conclusion{\psi}$ can be considered instead
of subproofs with one valent literal.  If the active literal is not valent, the corresponding
subproof will simply not be reintroduced later (i.e. the condition in line 28 of Algorithm \ref{algo:fullLUniv} will fail).

While verifying if a subproof could be univalent, some edges might be deleted. If a
subproof $\varphi_i$ has already been collected as univalent subproof with univalent literal
$\ell_i$ and the subproof $\varphi'$ being considered now has $\ell_i$ as active literal, the
corresponding incoming edges can be removed. Even if $\ell_i$ is valent for $\varphi'$, only
$\dual{\ell_i}$ would be introduced, and it would be resolved away when reintroducing
$\varphi_i$. The \FuncSty{delete} operation can be easily modified to remove both nodes and edges.

Algorithm \ref{algo:fullLUniv} sums up the previous remarks for an efficient implementation of
{\LowerUnivalents}. As noticed above, sometimes this algorithm may consider a subproof as univalent when it
is actually not. But as care is taken when reintroducing subproofs (at line \ref{line:full:testreintroduce}),
the resulting conclusion still subsumes the original.  The test that $\ell \in \Conclusion{\varphi}$
at line \ref{line:full:testactive} is mandatory since $\ell$ might have been deleted from
$\Conclusion{\varphi}$ by the deletion of previously collected subproofs.

\begin{algorithm}[pbt]
  \SetAlgoVlined
  \SetAlgoShortEnd

  \KwData {a proof $\psi$, compressed in place}
  \KwIn {a set $D_V$ of subproofs to delete}
  \KwIn {a set $D_E$ of edges to delete}
%  \KwOut{the proof $\psi$ compressed in place}
  \BlankLine

  \SetKw{Push}{push}
  \SetKw{Pop} {pop}
  \SetKw{Add} {add}
  \SetKw{Rep} {replace}

  \SetKwData{Activ}{ActiveLiterals}

  \Univ $\leftarrow \varnothing$ \;
  $\Delta \leftarrow \varnothing$ \;
  \BlankLine

  \For{every subproof $\varphi$, in a top-down traversal of $\psi$ }{

    \tcp{The deletion part.}
    \If{$\varphi$ is not an axiom}{
      \Let{$\varphi = \varphi_L \odot_\ell \varphi_R$} \;
      \uIf{ $\varphi_L \in D_V$ or $\pedge{\varphi}{\dual{\ell}}{\varphi_L} \in D_E$ }{
        \uIf{ $\pedge{\varphi}{\ell}{\varphi_R} \in D_E$ }{
          \Add $\varphi$ to $D_V$ \;
        }
        \Else{
          \Rep $\varphi$ by $\varphi_R$ \;
        }
      }
      \ElseIf{ $\varphi_R \in D_V$ or $\pedge{\varphi}{\dual{\ell}}{\varphi_R} \in D_E$ }{
        \uIf{ $\pedge{\varphi}{\ell}{\varphi_L} \in D_E$ }{
          \Add $\varphi$ to $D_V$ \;
        }
        \Else{
          \Rep $\varphi$ by $\varphi_L$ \;
        }
      }
    }
    \BlankLine
    
    \tcp{Test whether $\varphi$ is univalent.}
    \Activ $\leftarrow \varnothing$ \;
    \For{each incoming edge $e = \n \xrightarrow{\ell} \raiz{\varphi}$, $e \notin D_E$ }{
      \uIf{$\dual{\ell} \in \Delta$}{
        \Add $e$ to $D_E$ \;
      }
      \ElseIf{$\ell \notin \Delta$, $\ell \in \Conclusion{\varphi}$ \label{line:full:testactive}
              and $\ell \notin \Conclusion{\psi}$ }{
        \Add $\ell$ to \Activ \;
      }
    }

%    \BlankLine
    \If{\Activ $= \{\ell\}$ and $\Conclusion{\varphi} \subseteq \Delta \cup \{\ell\}$ }{
      \Push $\dual{\ell}$ onto $\Delta$ \;
      \Push $\varphi$     onto \Univ  \;
    }
  }
  \BlankLine

  \tcp{Reintroduce lowered subproofs.}
  \While{\Univ $\neq \varnothing$}{
    $\varphi \leftarrow$ \Pop from \Univ \;
    $\ell \leftarrow$ \Pop from $\Delta$ \;
    \If{$\ell \in \Conclusion{\psi}$ \label{line:full:testreintroduce}  }{
      \Rep $\psi$ by $\varphi \odot_\ell \psi$ \;}
  }

  \caption{Optimized {\LowerUnivalents} as an enhanced \texttt{delete}}
  \label{algo:fullLUniv}
\end{algorithm}


Every node in a proof $\langle V, E, \Gamma \rangle$ has exactly two outgoing edges unless it is the
root of an axiom. Hence the number of axioms is $|V| - \frac{1}{2}\,|E|$ and because there is at
least one axiom, the average number of active literals per node is strictly less than two.
Therefore, if {\LowerUnivalents} is implemented as an improved recursive \FuncSty{delete}, its time
complexity remains linear, assuming membership of literals to the set $\Delta$ is computed in constant
time.

\begin{proposition} \label{prop:compression}
Given a proof $\psi$,
{\LowerUnits\unskip\FuncSty{(}$\psi$\FuncSty{)}}
has at least as many nodes as 
{\LowerUnivalents\unskip\FuncSty{(}$\psi$\FuncSty{)}}
if there are no two units in $\psi$ with the same conclusion.
\end{proposition}

\begin{proof}
A unit $\varphi$ has exactly one active literal $\ell$. Therefore $\varphi$ is collected by
{\LowerUnivalents} unless $\dual{\ell} \in \Delta$ or $\ell \in \Delta$. If $\dual{\ell} \in \Delta$
all the incoming edges to $\raiz{\varphi}$ are deleted. If $\ell \in \Delta$, every edge
$\n \xrightarrow{\dual{\ell}} \n'$ where $\n$ is on a path from $\raiz{\psi}$ to $\raiz{\varphi}$
is deleted.
%coming from a descendent of $\raiz{\varphi}$ and labeled by $\dual{\ell} are deleted.
In particular, for every edge $\n \xrightarrow{\ell} \raiz{\varphi}$ the edge $\n
\xrightarrow{\dual{\ell}} \n'$ is deleted.  Moreover, as $\ell$ is the only literal of $\varphi$'s
conclusion, $\varphi$ is propagated down the proof until the univalent subproof with valent literal
$\dual{\ell}$ is reintroduced. \qed
\end{proof}

In the case where there are at least two units with the same conclusion in $\psi$, the
compressed proof depends on the order in which the units are collected. For both algorithms, only one of these units appears in the compressed proof.



\section{A Simpler First-Order LowerUnits}
\label{sec:SimpleFOLU}

ToDo by Jan



\section{Experiments} \label{sec:exp}

ToDo by Jan

{\LowerUnits} has been implemented as a prototype\footnote{Source code available at \url{https://github.com/jgorzny/Skeptik}} in the functional programming language Scala\footnote{\url{http://www.scala-lang.org/}} as part of the \skeptik
 library\footnote{\url{https://github.com/Paradoxika/Skeptik}}. {\LowerUnits} has been implemented as a
 recursive \FuncSty{delete} improvement.

%number to change - Jan
The algorithm has been applied to {\bf 308} proofs produced by the {\SPASS}\footnote{\url{http://www.verit-solver.org/}} theorem prover on unsatisfiable benchmarks from the TPTP Problem Library\footnote{\url{http://www.cs.miami.edu/~tptp/}}. The proofs used were restricted to those which could be solved within 300 seconds by {\SPASS} on the Euler Cluster at the University of Victoria\footnote{\url{https://rcf.uvic.ca/euler.php}} using only the contraction and unifying resolution inference rules.

For each proof $\psi$ (with the result of {\LowerUnits} applied to the proof denoted by $\alpha(\psi)$), the time to compress the proof ($t(\psi)$), the compression ratio ($(|\psi|-|\alpha(\psi)|)/|\psi|$), the resolution compression ratio  ($(|\psi|_R-|\alpha(\psi)|_R)/|\psi|_R$), the compression speed ($(|\psi|-|\alpha(\psi)|)/t(\psi)$), and resolution compression speed ($(|\psi|_R-|\alpha(\psi)|_R)/t(\psi)$) were measured\footnote{The raw data is available at \url{https://docs.google.com/spreadsheets/d/1F1-t2OuhypmTQhLU6yTj42aiZ5CqqaZvhVvOzeFgn0k/edit\#gid=1182923972}}, where $|\psi|_R$ indicates the number of resolution inference rules in the proof $\psi$.


The experiments were executed on a laptop (2.8GHz Intel Core i7 processor with 4 GB of RAM (1333MHz DDR3) available to the Java Virtual Machine), and the prototype implementation performed well on this system. Figure \ref{} shows the compression time $t(\psi)$ for each proof, sorted by proof length, and figure \ref{} (respectively figure \ref{}) shows the compression speed (respectively resolution compression speed) for each proof, also sorted by proof length.




\section{Conclusions and Future Work}

{\LowerUnivalents}, the algorithm presented here, has been shown in the previous section to compress
more than {\LowerUnits}. This is so because, as demonstrated in Proposition \ref{prop:compression}, the
set of subproofs it lowers is always a superset of the set of subproofs lowered by {\LowerUnits}. It might
be possible to lower even more subproofs by finding a characterization of (efficiently) lowerable subproofs
broader than that of univalent subproofs considered here. This direction for future work promises to be challenging, though, as evidenced by the non-triviality of the optimizations discussed in Section \ref{sec:LUniv} for obtaining a linear-time implementation of {\LowerUnivalents}.



As discussed in Section \ref{sec:LUnivRPI}, the proposed algorithm can be embedded in the deletion traversal of other algorithms.  As
an example, it has been shown that the combination of {\LowerUnivalents} with {\RPI}, compared to
the sequential composition of {\LowerUnits} after {\RPI}, results in a better compression ratio with
only a small processing time overhead (Figure \ref{fig:LUnivRPI}). Other compression algorithms that also have a subproof
deletion or reconstruction phase (e.g. \ReduceReconstruct) could probably benefit from being
combined with {\LowerUnivalents} as well.

%\vspace{-10pt}
%\paragraph{Acknowledgments:}



\bibliographystyle{splncs}
\bibliography{biblio}


\end{document}

% vim: tw=100
