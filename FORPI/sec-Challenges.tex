
\section{First-Order Challenges}\label{sec:Challenges}


In this section, we describe challenges that have to be overcome in order to successfully adapt {\RPI} to the first-order case. The first example illustrates the need to take unification into account. The other two examples discuss complex issues that can arise when unification is taken into account in a naive way.

%straightforward example
\begin{example}\label{ex:unif} 
Consider the following proof $\psi$. When computed as in the propositional case, the safe literals for $\eta_3$ are $\{ \vdash q(c), ~ p(a,X)\}$.
%$\eta_1$'s literal is unifiable with $p(a,X)$, which is inherited from $\eta_3$'s safe literals. 
%Thus the proof can be regularized by recycling $\eta_1$.
%$\eta_1$ and $\eta_5$ and these two literals are unifiable. Further, the safe literals for $\eta_1$ includes $\eta_5$. Thus the proof can be regularized by recycling $\eta_1$.

\begin{scriptsize}
\begin{prooftree}
\def\e{\mbox{\ $\vdash$\ }}
\AxiomC{$\eta_1$: \e $p(W,X)$}
\AxiomC{$\eta_2$: $p(W,X)$ \e $q(c)$}
\BinaryInfC{$\eta_3$: \e $q(c)$}
\AxiomC{$\eta_4$: $q(c)$ \e $p(a,X)$}
\BinaryInfC{$\eta_5$: \e $p(a,X)$}
\AxiomC{$\eta_6$: $p(Y,b)$ \e }
\BinaryInfC{$\psi$: $\bot$}
\end{prooftree}
\end{scriptsize}

\noindent
As neither of $\eta_3$'s pivots is syntactically equal to a safe literal, the propositional {\RPI} algorithm would not change $\psi$. However, $\eta_3$'s left pivot $p(W,X)\in \eta_1$ is unifiable with the safe literal $p(a,X)$. Regularizing $\eta_3$, by deleting the edge between $\eta_2$ and $\eta_3$ and replacing $\eta_3$ by $\eta_1$, leads to further deletion of $\eta_4$ (because it is not resolvable with $\eta_1$) and finally to the much shorter proof below.

\begin{footnotesize}
\begin{prooftree}
\def\e{\mbox{\ $\vdash$\ }}
\AxiomC{$\eta_1$: \e$p(W,X)$}
\AxiomC{$\eta_6$: $p(Y,b)$\e}
\BinaryInfC{$\psi'$: $\bot$}
\end{prooftree}
\end{footnotesize}

\end{example}

\noindent
Unlike in the propositional case, where a pivot must be syntactically equal to a safe literal for regularization to be possible, the example above suggests that, in the first-order case, it might suffice that a pivot be unifiable with a safe literal. However, there are cases, as shown in the example below, where mere unifiability is not enough and greater care is needed.

%unification necessary example
\begin{example}\label{ex:pairwise}

Again, the safe literals for $\eta_3$, when computed as in the propositional case, are $\{ \vdash q(c), ~ p(a,X)\}$, and as the pivot $p(a,c)$ is unifiable with the safe literal $p(a,X)$, $\eta_3$ appears to be a candidate for regularization. 

\begin{scriptsize}
\begin{prooftree}
\def\e{\mbox{\ $\vdash$\ }}
\AxiomC{$\eta_1$: \e $p(a,c)$}
\AxiomC{$\eta_2$: $p(a,c)$ \e $q(c)$}
\BinaryInfC{$\eta_3$: \e $q(c)$}
\AxiomC{$\eta_4$: $q(c)$ \e $p(a,X)$}
\BinaryInfC{$\eta_5$: \e $p(a,X)$}
\AxiomC{$\eta_6$: $p(Y,b)$ \e }
\BinaryInfC{$\psi$: $\bot$}
\end{prooftree}
\end{scriptsize}


\noindent
However, if we attempt to regularize the proof, the same series of actions as in Example \ref{ex:unif} would 
require resolution between $\eta_1$ and $\eta_6$, which is not possible.
%result in the following resolution, which cannot be completed.
%\begin{footnotesize}
%\begin{prooftree}
%\def\e{\mbox{\ $\vdash$\ }}
%\AxiomC{$\eta_1$: \e$p(a,c)$}
%\AxiomC{$\eta_6$: $p(Y,b)$\e}
%\BinaryInfC{$\psi'$: ??}
%\end{prooftree}
%\end{footnotesize}

\end{example}

\noindent
One way to prevent the problem depicted above would be to require the pivot to be not only unifiable but in fact more general than a safe literal. A weaker (and better) requirement is possible, however, as defined below.

\begin{definition}
\label{prop:pair}
Let $\eta$ be a node with pivot $\ell'$ unifiable with safe literal $\ell$ which is resolved against literals $\ell_1$, \ldots, $\ell_n$ in a proof $\psi$. $\eta$ is said to satisfy the \emph{pre-regularization unifiability property} in $\psi$ if $\ell_1$,\ldots,$\ell_n$, and $\dual{\ell'}$ are unifiable.
\end{definition}

\noindent
One way to ensure this property is met is to slightly modify the notion of safe literals, by applying the unifier of the resolution step to the each pivot before adding it to the safe literals (cf. algorithm 3, lines 8 and 10). In the case of Example \ref{ex:pairwise}, this would result in $\eta_3$ having the safe literals $\{ \vdash q(c),~p(a,b)\}$, where clearly the pivot $p(a,c)$ in $\eta_1$ is not safe.

%TODO: continue from here
%extra check example
\begin{example}\label{ex:unifcheck}
Satisfying the pre-regularization unifiability property is not sufficient. Consider the proof $\psi$ below. After collecting the safe literals, $\eta_3$'s safe literals are $\{q(T,V),p(c,d)\vdash q(f(a,e),c)\}$.

\begin{scriptsize}
\begin{prooftree}
\def\e{\mbox{\ $\vdash$\ }}
\AxiomC{$\eta_8$: $q(f(a,e),c)\e$\hspace{-1cm}}
\AxiomC{$\eta_6$: $\e p(c,d)$\hspace{-4.5cm}}
\AxiomC{$\eta_1$: $p(U,V)\e q(f(a,V),U)$}
\AxiomC{$\eta_2$: $q(f(a,X),Y),q(T,X)\e q(f(a,Z),Y)$}
\BinaryInfC{$\eta_3$: $p(U,V),Q(T,V)\e q(f(a,Z),U)$\hspace{-1cm}}
\AxiomC{\hspace{-1cm} $\eta_4$: $\e q(R,S)$}
\BinaryInfC{$\eta_5$: $p(U,V)\e q(f(a,Z),U)$}
\BinaryInfC{$\eta_7$: $\e q(f(a,Z),c)$}
\BinaryInfC{$\psi$: $\bot$}
\end{prooftree}
\end{scriptsize}

\noindent
$\eta_3$'s pivot $q(f(a,V),U)$ is unifiable to (and even more general than) the safe literal $q(f(a,e),c)$. Attempting to regularize $\eta_3$ would lead to the removal of $\eta_2$, the replacement of $\eta_3$ by $\eta_1$ and the removal of $\eta_4$ (because $\eta_1$ does not contain the pivot required by $\eta_5$), with $\eta_5$ also being replaced by $\eta_1$. Then resolution between $\eta_1$ and $\eta_6$ results in $\eta_7'$, which cannot be resolved with $\eta_8$, as shown below.

\begin{scriptsize}
\begin{prooftree}
\def\e{\mbox{\ $\vdash$\ }}
\AxiomC{$\eta_8$: $Q(f(a,e),c)\e$}
\AxiomC{$\eta_6$: $\e P(c,d)$}
\AxiomC{$\eta_1$: $P(U,V)\e Q(f(a,V),U)$}
\BinaryInfC{$\eta_7'$: $\e Q(f(a,d),c)$}
\BinaryInfC{$\psi'$: ??}
\end{prooftree}
\end{scriptsize}

\noindent
$\eta_1$'s literal $q(f(a, V), U)$, which would be resolved with $\eta_8$'s literal, was changed to $Q(f(a,d),c)$ due to the resolution between $\eta_1$ and $\eta_6$.

\end{example}


%ToDo: The following paragraph is not understandable. It must be made clearer. It must be made more formal (as a definition) and more mathematically precise (with the same level of detail as definition 2). It must explain \textbf{why} this additional check suffices.

\noindent
Thus we additionally require the following property be satisfied.
\begin{definition}
\label{prop:extracheck}
Let $\eta$ be a node with safe literals $\phi$ that is marked for regularization with parents $\eta_1$ and $\eta_2$, where 
%without loss generality 
$\eta_2$ is marked as a \texttt{deletedNode}
in a proof $\psi$. $\eta$ is said to satisfy the \emph{regularization unifiability property} in $\psi$ if there exists a substitution $\sigma$ such that $\eta_1\sigma \subseteq \phi$.
\end{definition}
%we ensure that the replacement parent is (possibly after unification) contained entirely in the safe literals. 
This property ensures that the remainder of the proof does not expect a variable in $\eta_1$ to be unified to different values simultaneously. This property is not necessary in the propositional case, as the replacement node would not change lower in the proof. 


%In order to avoid these scenarios, we perform an additional check during inference removal. The node $\eta*$ which will replace a resolution $\eta$ (because $\eta$ would have a deleted parent), must be entirely contained, via unification which modifies only $\eta^*$'s variables, in the safe literals of $\eta$. 
%%In this example, $\eta_1$ does not satisfy this property: in order to unify with $\eta_3$'s safe literals, it would be necessary to send $V\rightarrow Z$ due to $\eta_1$'s second literal, but leave $V$ unchanged due $\eta_1$'s first literal, which is not possible. This check is not necessary in the propositional case, as the replacement node would be contained exactly in the set of safe literals, and would not change lower in the proof.
%In this example, $\eta_1$ does not satisfy this property. This check is not necessary in the propositional case, as the replacement node would be contained exactly in the set of safe literals, and would not change lower in the proof.


\newcommand{\la}{\leftarrow}
\begin{algorithm}[bt]
\begin{footnotesize}
\SetKwInOut{Input}{input}\SetKwInOut{Output}{output}
\SetKwData{units}{unitsQueue}
\SetKwData{fixedUnits}{fixedUnitsQueue}

\Input{A first-order proof $\psi$}
\Output{A possibly less-irregular first-order proof $\psi'$}

\BlankLine

$\psi'$ $\la$ $\psi$\;
traverse $\psi'$ bottom-up and \ForEach{node $\eta$ in $\psi'$}{
   \If{$\eta$ is a resolvent node}{
     setSafeLiterals($\eta$) \;
     regularizeIfPossible($\eta$)
   }
  }
$\psi'$ $\la$ fix($\psi'$) \;
\Return {$\psi'$}\;
\caption{\label{algo:FORPI} \texttt{\FORPI}}
\end{footnotesize}
\end{algorithm}


%intersection example?