
\section{A Simpler First-Order LowerUnits}
\label{sec:SimpleFOLU}

%Recall example \ref{ex:ambig}. In order to avoid this, we introduce a proof rule that applies a substitution. So that we would get the following proof

\begin{tiny}
\begin{prooftree}
\def\e{\mbox{\ $\vdash$\ }}
\AxiomC{$\eta_1$: $p(U),r(U~V),r(V~U),q(V)$\e}
\UnaryInfC{$\eta_2$: $p(c),r(c~V),r(V~c),q(V)$\e}
\AxiomC{$\eta_3$: \e$r(X~c)$}
\BinaryInfC{$\eta_4$: $p(c),r(c~X),q(X)$\e}
\AxiomC{$\eta_5$: \e$r(W~V)$}
\BinaryInfC{$\eta_6$: $p(c),q(V)$\e}
\AxiomC{$\eta_7$: $p(Z)$\e$q(d)$}
\BinaryInfC{$\eta_8$: $p(c),p(Z)$\e}
\UnaryInfC{$\eta_9$: $p(c)$\e}
\AxiomC{$\eta_{10}$: \e$p(c)$}
\BinaryInfC{$\psi$: $\bot$}
\end{prooftree}
\end{tiny}

Now $r(V, c)$ appears in the first left resolvent, which was the left aux formula in the original proof. Thus, the implementation can find that formula, and choose it in order to resolve the ambiguous resolution, instead of guessing a formula from the left resolvent that unifies with the right resolvent, which might go wrong.\\

TODO: Explain where the sub came from.\\

TODO: define the rule formally here?\\

TODO: describe when the rule is invoked in the implementation\\

By performing a traversal to collect the units of a proof and then optimistically deleting units, some compression can often be achieved in linear time. By ignoring whether or not a unit satisfies Property \ref{prop:rootpair}, we can attempt to lower it, and should compression fail because deletions changed the substitutions to the point where contraction was not possible, we simply return the original proof. 

\begin{algorithm}[bt]
  \SetAlgoVlined
  \SetAlgoShortEnd
\SetKwFunction{check}{check}
  \KwIn {a proof $\psi$}
  \KwOut{a compressed proof $\psi^{\star}$}
  \KwData{a map $.'$: after line 4, it maps any $\varphi$ to \Del{$\varphi$, $D$}}
  \BlankLine

  \SetKwData{Units}{Units}

  \SetKw{Remove} {remove}
  \SetKw{Break} {break}

  \algolines{\Units $\leftarrow \varnothing$}{queue to store collected units}
  \BlankLine

  \For{every subproof $\varphi$, in a bottom-up traversal of $\psi$}{
    \lIf{$\varphi$ is a unit with more than one child and all literals of $\varphi$ are simultaneously unifiable}{enqueue $\varphi$ in \Units}
  }
  \BlankLine

    $\psi' \leftarrow $ \FuncSty{simple-fo-delete}$(\psi,\Units)$ \;
    \BlankLine

    \tcp{Reintroduce units}
    

    $\psi^{\star} \leftarrow \psi'$ \;
    \For{every unit $\varphi$ in \Units}{
        \Let{$\sigma$ be the unifier of $\rho(\psi^\star)$'s literals that contracts $\rho(\psi^\star)$ as much as possible} \;
        \Let{$c$ be the literals contracted by $\sigma$} \;
       \If{$ \con{\psi^{\star}}{c}{\sigma}$ and $\varphi$ can be resolved} {
        $\psi^{\star} \leftarrow \con{\psi^{\star}}{c}{\sigma} \res{\ell^c}{}{\ell}{} \varphi$ \;
        }\Else { \Return $\psi$}
      
    }
  
    

  \caption{\SFOLowerUnits}
  \label{algo:simpleFOLU}
\end{algorithm}

Algorithm \ref{algo:simpleFOLU} works similarly to the propositional algorithm.  It first performs a bottom up traversal to collect units which satisfy Property \ref{prop:pair}, adding those that do to a queue (line 3). It then attempts to re-introduce all the removed units at the bottom of the proof, where it attempts to compress the literals that would be resolved away by each unit (lines 6-15). In order to avoid traversing the proof to find literals resolved away from by each unit again after the deletion of every potential unit (as is done in Algorithm \ref{algo:FOLU}), we use a modified \FuncSty{delete} function, called \FuncSty{simple-fo-delete}, which is the same as Algorithm \ref{algo:del} except with line 6 changed to the following:

   \lIf{$~\varphi'_L \in D~$}{ 
     \Return{$(\rho(\varphi'_L) \sigma_R)$} 
    }
    \lElseIf{$\varphi'_R \in D$}{ 
      \Return{$(\rho(\varphi'_R) \sigma_R)$}  
    }

\FuncSty{simple-fo-delete} is designed to reduce the complexity of tracking literals. \FuncSty{simple-fo-delete} behaves much more closely to the propositional case and requires none of the additional data structures required by \FuncSty{fo-delete}. In this function, when a unit node is returned, instead of returning the opposite node (respectively $\psi_L'$ or $\psi_R'$, line 6) in the resolution (which is done in the propositional case), or tracking the literals (which is done in \FuncSty{fo-delete}), we return the opposite node with $\sigma_L$ (respectively $\sigma_R$) applied to it. In this way, the literals not resolved with the unit will look like they would have in the original proof, and the literals which were not resolved due to the deletion looks like it is syntactically equal with the unit literal at this stage. The fact that the other literals look like they did in the original proof is key: now resolution in the compressed proof can use the old literals, which should appear as they before, and not worry about choosing the wrong literal in case of ambiguous resolution. 

%A negative side-effect of this is that we may end up grounding literals, and having to carry these forms of each literal forward, which may increase the character length of the clause, though not the number of nodes in the proof.

However, by modifying delete in this manner we can longer guarantee that Property \ref{prop:rootpair} is satisfied. The appearance of literals that were to be resolved away from a unit clause may have changed, preventing completion of the proof. If this happens {\SFOLowerUnits} will attempt to re-introduce this node and fail, returning the original input proof (line 12).As a result, some proofs that can be compressed are returned unmodified, but those that do not require this additional property can be compressed much more quickly.

Alternatively, if additional unifiers do not modify the literals resolved away by a unit, those literals will still be simultaneously unifiable, and any contraction that reduces the length by as much as possible, such as one in line 7, will reduce these literals to a single literal, which will be unifiable with the unit. 



%\begin{algorithm}[bt]
  \SetAlgoVlined
  \SetAlgoShortEnd
  \KwIn{a proof $\varphi$}
  \KwIn{$D$ a set of subproofs}
  \KwOut{a proof $\varphi'$ obtained by deleting the subproofs in $D$ from $\varphi$}
  \BlankLine

  \newcommand{\fixL}{\ensuremath{\varphi'_L}}
  \newcommand{\fixR}{\ensuremath{\varphi'_R}}

  \lIf{$\varphi \in D$ or $\raiz{\varphi}$ has no premises}{\Return{$\varphi$}}
  \BlankLine

  \Else{$\varphi = \varphi_L \res{\ell_L}{\sigma_L}{\ell_R}{\sigma_R} \varphi_R$\;
    $\varphi'_L \leftarrow $ \Rec{$\varphi_L$,$D$} \;
    $\varphi'_R \leftarrow $ \Rec{$\varphi_R$,$D$} \;
    \BlankLine

    \lIf{$\varphi'_L \in D$}{ 
      \Return{$($\fixR $\sigma_R)$} 
    }
    \lElseIf{$\varphi'_R \in D$}{ 
      \Return{$($\fixL $\sigma_R)$}  
    }
    \BlankLine


    \lElse{ 
      \Return{ \fixL~$\res{\ell_L}{}{\ell_R}{}$~\fixR}
    }
  }



  \caption[.]{\FuncSty{simple-fo-delete}}
  \label{algo:sfodel}
\end{algorithm}


