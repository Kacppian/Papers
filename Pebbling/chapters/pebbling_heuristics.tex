\section{Heuristics}
\label{sec:heuristics}

The presented algorithms are parametrized by a heuristic, selecting one node $\n$ out of a set of nodes $N$.
%Heuristics are used in both pebbling algorithms to choose one node out of a set $N$. 
For Top-Down Pebbling, $N$ is the set of pebbleable nodes, and for Bottom-Up Pebbling, $N$ is the set of unprocessed premises of a node. 

\begin{definition}[Heuristic]

Let $\varphi$ be a proof with nodes $V$.
A \emph{heuristic} $h$ for $\varphi$ is a totally ordered set $S_h$ together with a \emph{node evaluation} function $e_h: V \rightarrow S_h$.
The \emph{choice} of the heuristic for a set $N \subseteq V$ is some $v \in N$ such that $v = \mathit{argmax}_{v \in N} e_h(v)$
\end{definition}

The $\mathit{argmax}$ of $e_h(v)$ is not unique in general.
In practice, we simply use another heuristic to decide ties and eventually have to decide upon some trivial criteria as for example address in memory.
We do not elaborate on the results of using different heuristics to decide ties.

In the following paragraphs, we present and motivate heuristics that rank nodes upon structural characteristics proofs.

\subsection{Number of Children Heuristic (``$Ch$'')}
\label{sec:children}
The \emph{Number of Children} heuristic uses the number of children of a node $v$ as evaluation function, i.e. $e_h(v) = |\Children{v}{\varphi}|$ and $S_h = \mathbb{N}$.
The intuitive motivation for this heuristic is that nodes with many children will require many pebbles, and subproofs containing nodes with many children will tend to be more spacious.
Example \ref{example:hardfirst} shows the idea behind pebbling spacious subproofs early.

\subsection{Last Child Heuristic (``$Lc$'')}
\label{sec:lastchild}

As discussed in Section \ref{sec:pebbling-game} in the proof of Theorem \ref{theorem:canonical}, the best moment to unpebble a node $v$ is as soon as its last child w.r.t. a topological order $\prec$ is pebbled. 
This insight is used for the \emph{Last Child} heuristic that chooses nodes that are last children of other nodes. 
Pebbling a node that allows another one to be unpebbled is always a good move. 
The current number of used pebbles (after pebbling the node and unpebbling one of its premises) does not increase.
It might even decrease, if more than one premise can be unpebbled.
For determining the number of premises of which a node is the last child, the proof has to be traversed once, before constructing the new order, using some topological order $\prec$.
Before the traversal, $e_h(v) = 0$ for every node $v$. 
During the traversal $e_h(v)$ is incremented by 1, if $v$ is the last child of the currently processed node w.r.t. $\prec$. 
For this heuristic $S_h = \mathbb{N}$.

To some extent, this heuristic is paradoxical: $v$ may be the last child of a node $v'$ according to $\prec$, but pebbling it early may result in another topological order $\prec^*$ according to which $v$ is not the last child of $v'$.
Nevertheless, often the proof structure ensures that a node is the last child of another node irrespective of the topological order. 

%An example is shown in Figure \ref{fig:forcedLC}, where the dashed line denotes a recursive predecessor relationship and the bottommost node is the last child of the top right node in every topological order.

%\begin{figure}
	%\centering{
	%\begin{tikzpicture}[node distance=1cm]
		%\proofnodeBW{n4};
		%\proofnodeBW[above left of = n4]{n3};
		%
		%%\proofnodeBW[above left of = n3]{n1};
		%
		%\proofnodeBW[above right of = n3]{n2};
		%%\withchildrenBW{n3}{n1}{n2};
		%%\drawchildren{n4}{n3}{n2};
		%%\node[above left of=n3] (emptynode) {};
		%%\draw[->,thick,cap=round,dashed] (n3) to (emptynode);
		%\draw[->,thick,cap=round,dashed] (n3) to (n2);
		%\draw[->,thick,cap=round] (n4) to (n2);
		%\draw[->,thick,cap=round,dashed] (n4) to (n3);
		%
	%\end{tikzpicture}
	%}
	%\caption{ }
	%\label{fig:forcedLC}
%\end{figure}


\subsection{Node Distance Heuristic (``$Dist(r)$'')}
\label{sec:distance}

In Example \ref{example:topdown} and Section \ref{sec:TDvsBU} it has been noted that Top-Down Pebbling may perform badly if nodes that are far apart are selected by the heuristic.
The \emph{Node Distance} heuristic prefers to pebble nodes that are close to pebbled nodes. It does this by calculating spheres with a radius up to the parameter $r$ around nodes.
The sphere $K_r^{G}(v)$ with radius $r$ around the node $v$ in the graph $G = (V,E)$ is defined as the set 
of nodes in $V$ that is connected to $v$ via at most $r$ undirected edges.
The heuristic uses the following functions based on the spheres:
\begin{align*}
	d(v) &\defeq 
	\begin{cases}
		-D &\text{where } D = min\{r \mid K_r^{G}(v)  \text{ contains}\\
			 & \hspace{3.3cm}  \text{a pebbled node}\}\\
		\infty & \text{if no such } D \text{ exists} 
	\end{cases}\\
	s(v) &\defeq |K_{-d(v)}^G(v)|\\
	l(v) &\defeq max_{\prec}K_{-d(v)}^G(v)\\
	e_h(v) &\defeq (d(v),s(v),l(v))
\end{align*}
where $\prec$ denotes the order of previously pebbled nodes.
So $S_h = \mathbb{Z} \cup \{\infty\} \times \mathbb{N} \times P$ together with the lexicographic order using, respectively, the natural smaller relation $<$ on $\mathbb{N}$ and $\mathbb{Z}$, where $\infty$ is an element that is bigger than all others, and $\prec$ on $N$. The spheres $K_r(v)$ can grow exponentially in $r$. Therefore the maximum radius has to be kept small.

\subsection{Decay Heuristics (``$Dc(h_u,\gamma,d,c)$'') }
\label{sec:decay}
Decay heuristics denote a family of meta heuristics. 
The idea is to not only use the evaluation of a single node, but also to include the evaluations of its premises.
Such a heuristic has four parameters: an underlying heuristic $h_u$ defined by an evaluation function $e_u$ together with a well ordered set $S_u$, a decay factor $\gamma \in \mathbb{R}^+ \cup \{0\}$, a recursion depth $d \in \mathbb{N}$ and a combining function $c: S_u^n \rightarrow S_u$ for $n \in \mathbb{N}$.
The resulting heuristic node evaluation function $e_h$ is defined recursively, using function $r$:
\begin{align*}
	r(v,0) &\defeq e_u(v) \\
	r(v,k) &\defeq e_u(v) + c(r(p_1,k-1), \ldots,r(p_n,k-1)) * \gamma\\
	& \text{ where } \Premises{v}{\varphi} = \{p_1,\ldots,p_n\}\\
	% & \text{ and } k \in \{1,\ldots,d\} \\
	%rec(\n,k) &\defeq com(h_u(\n), rec(p_1,k-1)*\gamma,\ldots,rec(p_{n-1},k-1) * \gamma)\\ %might be better - I will try this in the experiments
	e_h(v) &\defeq r(v,d) &
\end{align*}

