\section{Greedy Pebbling Algorithms}
\label{sec:algorithms}

Theorem \ref{theorem:enumeration} and the remarks in the end of Section \ref{sec:pebblingSAT} indicate that obtaining an optimal topological order either by enumerating topological orders or by encoding the problem as a satisfiability problem is impractical. 
This section presents two greedy algorithms that aim at finding good though not necessarily optimal topological orders. 
They are both parameterized by some heuristic described in Section \ref{sec:heuristics}, but differ in the traversal direction in which the algorithms operate on proofs.

\subsection{Top-Down Pebbling}

Top-Down Pebbling (Algorithm \ref{algo:TDpebbling}) constructs a topological order of a proof $\varphi$ by traversing it from its axioms to its root node.
This approach closely corresponds to how a human would play the Bounded Pebbling Game. 
A human would look at the nodes that are available for pebbling in the current round of the game, choose one of them to pebble and remove pebbles if possible.
Similarly the algorithm keeps track of pebblable nodes in a set $N$, initialized as $\Axioms{\varphi}$.
When a node $v$ is pebbled, it is removed from $N$ and added to the sequence representing the topological order. The children of $v$ that become pebbleable are added to $N$.
When $N$ becomes empty, all nodes have been pebbled once and a topological order has been found.


\begin{algorithm}[h]
  \KwIn{proof $\varphi$}
  \KwOut{sequence of nodes $S$ representing a topological order $\prec$ of $\varphi$}
	
	
	$S = ()$; \tcp*[f]{the empty sequence} \\
	
	$N = \Axioms{\varphi}$; \tcp*[f]{pebbleable nodes} \\
	
  \While{$N$ is not empty}{
    choose $v \in N$ heuristically\;
		$S = S ::: (v)$; \tcp*[f]{$:::$ is sequence concatenation}\\
		$N = N \setminus \{v\}$\;
		\tcp*[f]{check whether $c$ is now pebbleable}\\
		\For {\KwSty{each} $c \in \Children{v}{\varphi}$}{ 	
			\If{$\forall p \in \Premises{c}{\varphi}: p \in S$}
					{$N = N \cup \{c\}$\;}
					}
  }
	
	\Return $S$\;
	
  \caption[.]{\FuncSty{Top-Down Pebbling}}
  \label{algo:TDpebbling}
\end{algorithm}

Top-Down Pebbling often constructs pebbling strategies with high pebbling numbers regardless of the heuristic used.
The following example shows such a situation.

\begin{example}
\label{example:topdown}

\begin{figure}[tb]
	\makebox[.5\textwidth][c]{
	\begin{minipage}{.25\textwidth}
		\begin{tikzpicture}[node distance=\nodedistance]
			\proofnodeBW{root};
			\proofnodeBW[above left of = root,xshift=-\nodedistance]{n7};
			\proofnodeBW[above right of = n7,xshift=\nodedistance]{n6};
			\proofnodeBW[above left of = n7]{n3};
			\proofnodeBW[above right of = root,xshift=\nodedistance]{n12};
			\proofnodeBW[above right of = n12]{n10};
			\withchildrenBW{n3}{n1}{n2};
			\withchildrenBW{n10}{n8}{n9};
			\proofnodeBW[above right of = n2]{n4};
			\proofnodeBW[above left of = n8]{n5};
			\drawchildren{n6}{n4}{n5};
			\withchildrenBW{n4}{e1}{e2};
			\withchildrenBW{n5}{e3}{e4};
			\drawchildren{root}{n7}{n12};
			\drawchildren{n7}{n3}{n6};
			\drawchildren{n12}{n6}{n10};
			\blacknode{n3};
			\node [yshift = 1.5mm] (cap1) at (n1.north) {\small{1}};
			\node [yshift = 1.5mm] (cap2) at (n2.north) {\small{2}};
			\node [yshift = 1.5mm] (cap3) at (n3.north) {\small{3}};
			\node [yshift = 1.5mm] (cap4) at (e1.north) {\small{4?}};
			\node [yshift = 1.5mm] (cap4) at (e2.north) {\small{4?}};
			\node [yshift = 1.5mm] (cap5) at (e3.north) {\small{4?}};
			\node [yshift = 1.5mm] (cap5) at (e4.north) {\small{4?}};
			\node [yshift = 1.5mm] (cap6) at (n8.north) {\small{4?}};
			\node [yshift = 1.5mm] (cap7) at (n9.north) {\small{4?}};
		\end{tikzpicture}
	\end{minipage}%
	\begin{minipage}{.25\textwidth}
		\begin{tikzpicture}[node distance=\nodedistance]
			\proofnodeBW{root};
			\proofnodeBW[above left of = root,xshift=-\nodedistance]{n7};
			\proofnodeBW[above right of = n7,xshift=\nodedistance]{n6};
			\proofnodeBW[above left of = n7]{n3};
			\proofnodeBW[above right of = root,xshift=\nodedistance]{n12};
			\proofnodeBW[above right of = n12]{n10};
			\withchildrenBW{n3}{n1}{n2};
			\withchildrenBW{n10}{n8}{n9};
			\proofnodeBW[above right of = n2]{n4};
			\proofnodeBW[above left of = n8]{n5};
			\drawchildren{n6}{n4}{n5};
			\withchildrenBW{n4}{e1}{e2};
			\withchildrenBW{n5}{e3}{e4};
			\drawchildren{root}{n7}{n12};
			\drawchildren{n7}{n3}{n6};
			\drawchildren{n12}{n6}{n10};
			\blacknode{n3};
			\blacknode{n8};
			\node [yshift = 2mm] (cap1) at (n1.north) {\small{1}};
			\node [yshift = 2mm] (cap2) at (n2.north) {\small{2}};
			\node [yshift = 2mm] (cap3) at (n3.north) {\small{3}};
			\node [yshift = 2mm] (cap8) at (n8.north) {\small{4}};
		\end{tikzpicture}
	\end{minipage}%
	}
	\makebox[.5\textwidth][c]{
	\begin{minipage}{.25\textwidth}
	\begin{tikzpicture}[node distance=\nodedistance]
			\proofnodeBW{root};
			\proofnodeBW[above left of = root,xshift=-\nodedistance]{n7};
			\proofnodeBW[above right of = n7,xshift=\nodedistance]{n6};
			\proofnodeBW[above left of = n7]{n3};
			\proofnodeBW[above right of = root,xshift=\nodedistance]{n12};
			\proofnodeBW[above right of = n12]{n10};
			\withchildrenBW{n3}{n1}{n2};
			\withchildrenBW{n10}{n8}{n9};
			\proofnodeBW[above right of = n2]{n4};
			\proofnodeBW[above left of = n8]{n5};
			\drawchildren{n6}{n4}{n5};
			\withchildrenBW{n4}{e1}{e2};
			\withchildrenBW{n5}{e3}{e4};
			\drawchildren{root}{n7}{n12};
			\drawchildren{n7}{n3}{n6};
			\drawchildren{n12}{n6}{n10};
			\blacknode{n3};
			\blacknode{n10};
			\blacknode{n4};
			\blacknode{n5};
			\blacknode{e3};
			\blacknode{e4};
			\node [yshift = 2mm] (cap1) at (n1.north) {\small{1}};
			\node [yshift = 2mm] (cap2) at (n2.north) {\small{2}};
			\node [yshift = 2mm] (cap3) at (n3.north) {\small{3}};
			\node [yshift = 2mm] (cap4) at (n8.north) {\small{4}};
			\node [yshift = 2mm] (cap5) at (n9.north) {\small{5}};
			\node [yshift = 2mm] (cap6) at (n10.north) {\small{6}};
			\node [yshift = 2mm] (cap7) at (e1.north) {\small{7}};
			\node [yshift = 2mm] (cap8) at (e2.north) {\small{8}};
			\node [yshift = 2mm] (cap8) at (n4.north) {\small{9}};
			\node [yshift = 2mm] (cap7) at (e3.north) {\small{10}};
			\node [yshift = 2mm] (cap8) at (e4.north) {\small{11}};
			\node [yshift = 2mm] (cap8) at (n5.north) {\small{12}};
		\end{tikzpicture}
	\end{minipage}%
	}
	\caption{Top-Down Pebbling}
	\label{fig:TDP}
\end{figure}

Consider the graph shown in Figure \ref{fig:TDP} and suppose that Top-Down Pebbling has already pebbled the initial sequence of nodes $(1,2,3)$. 
For a greedy heuristic that only has information about pebbled nodes, their premises and children, all nodes marked with $4?$ are considered equally worthy to pebble next.
Suppose the node marked with $4$ in the middle graph is chosen to be pebbled next.
Subsequently, pebbling $5$ opens up the possibility to remove a pebble after the next move, which is to pebble $6$.
After that only the middle subgraph has to be pebbled. 
No matter in which order this is done, the strategy will use six pebbles at some point. 
One example sequence and the point where six pebbles are used are shown in the rightmost picture in Figure \ref{fig:TDP}.
However the pebbling number of this proof is five.

\label{example:TDPIssue}
\end{example}

\subsection{Bottom-Up Pebbling}

Bottom-Up Pebbling (Algorithm \ref{algo:BUpebbling}) constructs a topological order of a proof $\varphi$ while traversing it from its root node $\n$ to its axioms. 
The algorithm constructs the order by visiting nodes and their premises recursively. 
For every node $v$ the order in which the premises of $v$ are visited is decided heuristically. 
After visiting the premises, $v$ is added to the current sequence of nodes.
Since axioms do not have any premises, there is no recursive call for axioms and these nodes are simply added to the sequence. 
The recursion is started with the call \texttt{BUpebble($\varphi,\n,\emptyset,()$)}.
Since all proof nodes are ancestors of the root, the recursive calls will eventually visit all nodes once and a topological total order will be found.
Bottom-Up Pebbling corresponds to the apply function $\ap(.)$ defined in Section \ref{sec:resolution} with the addition of a visit order of the premises.
Also previously visited nodes are not visited again.

\SetKwFunction{KwVisit}{BUpebble}

%\begin{algorithm}[h]
  %\KwIn{proof $\varphi$ with root node $r$}
  %\KwOut{sequence of nodes $S$ representing a topological order $\prec$ of $\varphi$}
  %\BlankLine
%
	%$S = ()$\; \tcp*[f]{the empty sequence}\\
	%$V = \emptyset$\;
	%\Return \KwVisit{$\varphi$,$r$,$V$,$S$}\;
%
  %\caption[.]{Bottom-Up Pebbling}
  %\label{algo:BUpebbling}
%\end{algorithm}

\begin{algorithm}[h]
  \KwIn{proof $\varphi$}
	\KwIn{node $v$}
	\KwIn{set of visited nodes $D$} 
	\KwIn{initial sequence of nodes $S$}
  \KwOut{sequence of nodes}
	
	$D = D \cup \{v\}$; \\
	$N = \Premises{v}{\varphi} \setminus D$; \tcp*[f]{Visit only unprocessed premises} \\
	$S_1 = S$\;
	
  \While{$N$ is not empty}{
    choose $p \in N$ heuristically\;
		$N = N \setminus p$\;
		$S_1 = S_1 ::: Bottom-Up Pebbling(\varphi,p,D,S)$\; 
  }
	
	\Return $S_1 ::: (v)$\;
	
  \caption{Bottom-Up Pebbling}
	\label{algo:BUpebbling}
  
\end{algorithm}

%\newcommand{\nodedistance2}{0.6cm}
\begin{example}
Figure \ref{fig:BUP} shows part of an execution of Bottom-Up Pebbling on the same proof as presented in Figure \ref{fig:TDP}.
Nodes chosen by the heuristic, to be processed before the respective other premise, are marked dashed. 
Suppose that similarly to the Top-Down Pebbling scenario, nodes have been chosen in such a way that the initial pebbling sequence is $(1,2,3)$.
However, the choice of where to go next is predefined by the dashed nodes. 
Consider the dashed child of node $3$. 
Since $3$ has been completely processed, the other premise of its dashed child is visited next. 
The result is that the middle subgraph is pebbled with only one pebble placed on a node that does not belong to the subgraph.
In the Top-Down scenario there were two such external pebbles. 
At no point more than five pebbles will be used for pebbling the root node, which is shown in the bottom right picture of the figure. This is independent of the heuristic choices.

\begin{figure}[tb]
	\makebox[.5\textwidth][c]{
		\begin{minipage}{0.25\textwidth}
			\begin{tikzpicture}[node distance=\nodedistanceThree]
				\proofnodeBW{root};
				\proofnodeBW[above left of = root,xshift=-\nodedistanceThree]{n7};
				\proofnodeBW[above right of = n7,xshift=\nodedistanceThree]{n6};
				\proofnodeBW[above left of = n7]{n3};
				\proofnodeBW[above right of = root,xshift=\nodedistanceThree]{n12};
				\proofnodeBW[above right of = n12]{n10};
				\withchildrenBW{n3}{n1}{n2};
				\withchildrenBW{n10}{n8}{n9};
				\proofnodeBW[above right of = n2]{n4};
				\proofnodeBW[above left of = n8]{n5};
				\drawchildren{n6}{n4}{n5};
				\withchildrenBW{n4}{e1}{e2};
				\withchildrenBW{n5}{e3}{e4};
				\drawchildren{root}{n7}{n12};
				\drawchildren{n7}{n3}{n6};
				\drawchildren{n12}{n6}{n10};
				\waitingnode{n7};
				\waitingnode{root};
				\blacknode{n3};
				\node [yshift = 2mm] (cap1) at (n1.north) {\small{1}};
				\node [yshift = 2mm] (cap2) at (n2.north) {\small{2}};
				\node [yshift = 2mm] (cap3) at (n3.north) {\small{3}};
			\end{tikzpicture}
		\end{minipage}%
		\begin{minipage}{0.25\textwidth}
			\begin{tikzpicture}[node distance=\nodedistance]
				\proofnodeBW{root};
				\proofnodeBW[above left of = root,xshift=-\nodedistanceThree]{n7};
				\proofnodeBW[above right of = n7,xshift=\nodedistanceThree]{n6};
				\proofnodeBW[above left of = n7]{n3};
				\proofnodeBW[above right of = root,xshift=\nodedistanceThree]{n12};
				\proofnodeBW[above right of = n12]{n10};
				\withchildrenBW{n3}{n1}{n2};
				\withchildrenBW{n10}{n8}{n9};
				\proofnodeBW[above right of = n2]{n4};
				\proofnodeBW[above left of = n8]{n5};
				\drawchildren{n6}{n4}{n5};
				\withchildrenBW{n4}{e1}{e2};
				\withchildrenBW{n5}{e3}{e4};
				\drawchildren{root}{n7}{n12};
				\drawchildren{n7}{n3}{n6};
				\drawchildren{n12}{n6}{n10};
				\waitingnode{n7};
				\blacknode{n3};
				\waitingnode{n6};
				\waitingnode{root};
				\node [yshift = 2mm] (cap1) at (n1.north) {\small{1}};
				\node [yshift = 2mm] (cap2) at (n2.north) {\small{2}};
				\node [yshift = 2mm] (cap3) at (n3.north) {\small{3}};
			\end{tikzpicture}
		\end{minipage}%
		}
		\makebox[.5\textwidth][c]{
		\begin{minipage}{0.25\textwidth}
			\begin{tikzpicture}[node distance=\nodedistanceThree]
				\proofnodeBW{root};
				\proofnodeBW[above left of = root,xshift=-\nodedistanceThree]{n7};
				\proofnodeBW[above right of = n7,xshift=\nodedistanceThree]{n6};
				\proofnodeBW[above left of = n7]{n3};
				\proofnodeBW[above right of = root,xshift=\nodedistanceThree]{n12};
				\proofnodeBW[above right of = n12]{n10};
				\withchildrenBW{n3}{n1}{n2};
				\withchildrenBW{n10}{n8}{n9};
				\proofnodeBW[above right of = n2]{n4};
				\proofnodeBW[above left of = n8]{n5};
				\drawchildren{n6}{n4}{n5};
				\withchildrenBW{n4}{e1}{e2};
				\withchildrenBW{n5}{e3}{e4};
				\drawchildren{root}{n7}{n12};
				\drawchildren{n7}{n3}{n6};
				\drawchildren{n12}{n6}{n10};
				\blacknode{n7};
				\blacknode{n6};
				\waitingnode{root};
				\node [yshift = 2mm] (cap1) at (n1.north) {\small{1}};
				\node [yshift = 2mm] (cap2) at (n2.north) {\small{2}};
				\node [yshift = 2mm] (cap3) at (n3.north) {\small{3}};
				\node [yshift = 2mm] (cap4) at (e1.north) {\small{5}};
				\node [yshift = 2mm] (cap5) at (e2.north) {\small{4}};
				\node [yshift = 2mm] (cap6) at (n4.north) {\small{6}};
				\node [yshift = 2mm] (cap7) at (e3.north) {\small{7}};
				\node [yshift = 2mm] (cap7) at (e4.north) {\small{8}};
				\node [yshift = 2mm] (cap7) at (n5.north) {\small{9}};
				\node [yshift = 2mm] (cap7) at (n6.north) {\small{10}};
			\end{tikzpicture}
		\end{minipage}%
		\begin{minipage}{0.25\textwidth}
			\begin{tikzpicture}[node distance=\nodedistanceThree]
				\proofnodeBW{root};
				\proofnodeBW[above left of = root,xshift=-\nodedistanceThree]{n7};
				\proofnodeBW[above right of = n7,xshift=\nodedistanceThree]{n6};
				\proofnodeBW[above left of = n7]{n3};
				\proofnodeBW[above right of = root,xshift=\nodedistanceThree]{n12};
				\proofnodeBW[above right of = n12]{n10};
				\withchildrenBW{n3}{n1}{n2};
				\withchildrenBW{n10}{n8}{n9};
				\proofnodeBW[above right of = n2]{n4};
				\proofnodeBW[above left of = n8]{n5};
				\drawchildren{n6}{n4}{n5};
				\withchildrenBW{n4}{e1}{e2};
				\withchildrenBW{n5}{e3}{e4};
				\drawchildren{root}{n7}{n12};
				\drawchildren{n7}{n3}{n6};
				\drawchildren{n12}{n6}{n10};
				\blacknode{n7};
				\blacknode{n6};
				\blacknode{n8};
				\blacknode{n9};
				\blacknode{n10};
				\waitingnode{root};
				\waitingnode{n12};
				\node [yshift = 2mm] (cap1) at (n1.north) {\small{1}};
				\node [yshift = 2mm] (cap2) at (n2.north) {\small{2}};
				\node [yshift = 2mm] (cap3) at (n3.north) {\small{3}};
				\node [yshift = 2mm] (cap4) at (e1.north) {\small{5}};
				\node [yshift = 2mm] (cap5) at (e2.north) {\small{4}};
				\node [yshift = 2mm] (cap6) at (n4.north) {\small{6}};
				\node [yshift = 2mm] (cap7) at (e3.north) {\small{7}};
				\node [yshift = 2mm] (cap7) at (e4.north) {\small{8}};
				\node [yshift = 2mm] (cap7) at (n5.north) {\small{9}};
				\node [yshift = 2mm] (cap7) at (n6.north) {\small{10}};
				\node [yshift = 2mm] (cap7) at (n8.north) {\small{11}};
				\node [yshift = 2mm] (cap7) at (n9.north) {\small{12}};
				\node [yshift = 2mm] (cap7) at (n10.north) {\small{13}};
			\end{tikzpicture}
		\end{minipage}%
			}
		\caption{Bottom-Up Pebbling}
		\label{fig:BUP}
\end{figure}
\label{example:BUP}
\end{example}



\subsection{Remarks about Top-Down and Bottom-Up Pebbling} %or: Which way to go?
\label{sec:TDvsBU}

The experiments presented in Section \ref{sec:experiments} show that in practice, Bottom-Up Pebbling performs much better than Top-Down.
Example \ref{example:TDPIssue} shows two principles that result in pebbling strategies with small pebbling numbers and are likely to be violated by the Top-Down Pebbling algorithm.

Firstly, a pebbling strategy should make local choices.
By local choices we mean that it should pebble nodes that are close w.r.t. undirected edges in the graph to other pebbled nodes.
Such local choices allow to unpebble other nodes earlier and therefore keep the pebbling number low.
Bottom-Up Pebbling makes local choices by design, because premises are queued up and the second premise is visited as soon as possible.
Top-Down Pebbling does not have knowledge about the recursive structure of child nodes, therefore it is hard to make local choices.
The algorithm simply does not know which pebbleable nodes are close to other pebbled ones.

Secondly, pebbling strategies should process subproofs with a high pebbling number early.
Pebbling such subproofs late will result in other pebbles staying on nodes for a high number of rounds.
This likely results in increasing the overall pebbling number, as this adds extra pebbles to the already high pebbling number of the subproof.
The principle is more subtle than the first one, because pebbling one subproof can influence the number of pebbles used for another subproof in situations where nodes are shared between subproofs.
The principle is demonstrated in the following example.

\begin{example}
Figure \ref{fig:SpaciousFirst} shows a simple proof $\varphi$ with two subproofs $\varphi_0$ (left branch) and $\varphi_1$ (right branch). 
As shown in the leftmost diagram, assume $\pspace{\varphi_0}{\prec_0} = 4$ and $\pspace{\varphi_1}{\prec_1} = 5$, where $\prec_0$ and $\prec_1$ represent some topological order of the respective subproofs with the corresponding pebbling numbers.
After pebbling one of the subproofs, the pebble on its root node has to be kept there until the root of the other subproof is also pebbled. 
Only then the root node can be pebbled. 
Therefore, $\pspace{\varphi}{\prec} = \pspace{\varphi_j}{\prec_j} + 1$ where $\prec$ is obtained by first pebbling according to $\prec_{1-j}$, then by $\prec_{j}$ followed by pebbling the root.
Choosing to pebble the less spacious subproof $\varphi_0$ first results in $\pspace{\varphi}{\prec} = 6$, while pebbling the more spacious one first gives $\pspace{\varphi}{\prec} = 5$.

Note that this example shows a simplified situation. 
The two subproofs do not share nodes. 
Pebbling one of them does not influence the pebbling number of the other.

\begin{figure}[h]
	\usetikzlibrary{shapes.geometric}
	\tikzset{
    triangle/.style={
        draw,
        shape border rotate=180,
        regular polygon,
        regular polygon sides=3,
        node distance=\nodedistanceTwo,
        %minimum height=4em,
				minimum width= 1cm
			}
	}
	\makebox[.5\textwidth][c]{
	\begin{minipage}{.15\textwidth}
		\begin{tikzpicture}[node distance=\nodedistanceTwo]
			\node[circle, draw, anchor=mid](root) {?};
			\addchildrenBW{root}{n1}{n2};
			\node[triangle, above of = n1, yshift = -4mm]{4};
			\node[triangle, above of = n2, yshift = -4mm]{5};
			\whitenode{n1};
			\whitenode{n2};
			\drawchildren{root}{n1}{n2};
		\end{tikzpicture}
	\end{minipage}%
		\begin{minipage}{.15\textwidth}
		\begin{tikzpicture}[node distance=\nodedistanceTwo]
			\node[circle, draw, anchor=mid](root) {6};
			\addchildrenBW{root}{n1}{n2};
			\node[triangle, above of = n1, yshift = -4mm]{ };
			\node[triangle, above of = n2, yshift = -4mm]{5};
			\blacknode{n1};
			\whitenode{n2};
			\drawchildren{root}{n1}{n2};
		\end{tikzpicture}
	\end{minipage}%
		\begin{minipage}{.15\textwidth}
		\begin{tikzpicture}[node distance=\nodedistanceTwo]
			%\proofnodeBW{root};
			\node[circle, draw, anchor=mid](root) {5};
			\addchildrenBW{root}{n1}{n2};
			\node[triangle, above of = n1, yshift = -4mm]{4};
			\node[triangle, above of = n2, yshift = -4mm]{ };
			\whitenode{n1};
			\blacknode{n2};
			\drawchildren{root}{n1}{n2};
		\end{tikzpicture}
	\end{minipage}%
	}
	\caption{Spacious subproof first}
	\label{fig:SpaciousFirst}
\end{figure}
\label{example:hardfirst}
\end{example}
