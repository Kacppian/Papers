

\section{First-Order LowerUnits} \label{sec:FOLU}

The examples shown in the previous section indicate that there are two main challenges that need to be overcome in order to generalize \LowerUnits to the first-order case:
\begin{enumerate}
\item The deletion of a node changes literals. Since substitutions associated with the deleted node are not applied anymore, some literals become more general. Therefore, the reconstruction of the proof during deletion needs to take such changes into account.
\item Whether a unit should be collected for lowering must depend on whether the literals that were resolved with the unit's single literal are unifiable after they are propagated down to the bottom of the proof by the process of unit deletion. Only if this is the case, they can be contracted and the unit can be reintroduced in the bottom of the proof.
\end{enumerate}

\newcommand{\none}{\texttt{none}}

\noindent
Algorithm \ref{algo:fodel} overcomes the first challenge by keeping an additional map from old literals in the input proof to the corresponding more general changed literals in the output proof unders construction. This is done in lines 6 to 7. The correspondence can be computed by proper bookkeeping during deletion (e.g. by having data structures that preserve the positions of literals or by annotating literals with ids). In cases where, due to previous deletions above in the proof, no corresponding literal is available anymore, the special constant $\none$ is used. 

Not only the literals, but also the substitutions must change during deletion. While it would be in principle possible to keep track of such changes as well, it is simpler to search for new substitutions that result in a most general resolved atom. This is why substitutions are omitted in line 12. As a beneficial side-effect, we get possibly more general literals in the root clause of the output proof.


\clearpage

\SetKwFunction{Rec}{delete}
\SetKw{Let}{let}

\newcommand{\dl}[1]{#1^{\dagger}}

\begin{algorithm}[bt]
  \SetAlgoVlined
  \SetAlgoShortEnd
  \KwIn{a proof $\varphi$}
  \KwIn{$D$ a set of subproofs}
  \KwOut{a proof $\varphi'$ obtained by deleting the subproofs in $D$ from $\varphi$}
  \KwData{a map $.'$, initially empty, eventually mapping any $\xi$ to \Del{$\xi$, $D$}}
  \KwData{a map $\dl{.}$, initially empty, eventually mapping literals to changed literals}
  \BlankLine

  \newcommand{\fixL}{\ensuremath{\varphi'_L}}
  \newcommand{\fixR}{\ensuremath{\varphi'_R}}

  \lIf{$\varphi \in D$ or $\raiz{\varphi}$ has no premises}{\Return{$\varphi$}}
  \BlankLine

  \Else{
    \Let{$\varphi_L \res{\ell_L}{\sigma_L}{\ell_R}{\sigma_R} \varphi_R = \varphi$} \;
    $\varphi'_L \leftarrow $ \Rec{$\varphi_L$,$D$} \;
    $\varphi'_R \leftarrow $ \Rec{$\varphi_R$,$D$} \;
    \BlankLine

    \For{every $\ell$ in $\Conclusion{\varphi_L}$ or $\Conclusion{\varphi_R}$}{
      $\dl{\ell} \leftarrow$ the literal in $\Conclusion{\varphi'_L}$ or $\Conclusion{\varphi'_R}$ corresponding to $\ell$, otherwise $\none$ \;
    }

    \lIf{$\varphi'_L \in D$}{ 
      \Return{\fixR} 
    }
    \lElseIf{$\varphi'_R \in D$}{ 
      \Return{\fixL}  
    }
    \BlankLine

    \lElseIf{$\dl{\ell_L} = \none$}{ 
      \Return{\fixL} 
    }
    \lElseIf{$\dl{\ell_R} = \none$}{ 
      \Return{\fixR}  
    }
    \BlankLine

    \lElse{ 
      \Return{ \fixL~$\res{\dl{\ell_L}}{}{\dl{\ell_R}}{}$~\fixR}
    }
  }

  \caption[.]{\FuncSty{fo-delete}}
  \label{algo:fodel}
\end{algorithm}


The second challenge is much harder to overcome. In the propositional case, collecting units and deleting units can be done in two distinct phases (as shown in Algorithm \ref{algo:lu}). In the first-order case, on the other hand, these two phases seem to be so interlaced, that they appear to be in a deadlock: the decision to collect a unit to be lowered depends on what will happen with the proof after deletion, while deletion depends on knowing which units will be lowered. Therefore, there is no hope for a single-traversal first-order {\LowerUnits} algorithm generalizing \ref{algo:optLU}. The collection of units must be done in a bottom-up traversal (as in \ref{algo:LU}), so that when we have to test whether a unit $\eta$ can be lowered (in line 3), we can run the deletion algorithm to see what will happen with the root of the proof when we delete $\eta$ and all other units already collected below $\eta$. This unlocks the apparent deadlock, at the high cost of having to run the deletion algorithm once for every tested unit. The worst-case run-time complexity is, therefore, quadratic in the length of the input proof to be compressed.

\begin{algorithm}[bt]
  \SetAlgoVlined
  \SetAlgoShortEnd
  \KwIn {a proof $\psi$}
  \KwOut{a compressed proof $\psi^{\star}$}
  \KwData{a map $.'$: after line 4, it maps any $\varphi$ to \Del{$\varphi$, $D$}}
  \BlankLine

  ToDo \;

  \SetKwData{Units}{Units}
  \algolines{\Units $\leftarrow \varnothing$}{queue to store collected units}
  \BlankLine

  \For{every subproof $\varphi$, in a bottom-up traversal of $\psi$}{
    \lIf{$\varphi$ is a unit with more than one child}{enqueue $\varphi$ in \Units}
  }
  \BlankLine

  $\psi' \leftarrow $ \Del{$\psi$,$\Units$} \;
  \BlankLine

  \tcp{Reintroduce units}
  $\psi^{\star} \leftarrow \psi'$ \;
  \For{every unit $\varphi$ in \Units}{
    \Let{$\{\ell\} = \Conclusion{\varphi}$} \;
    \lIf{$\dual{\ell} \in \Conclusion{\psi'}$}{
    $\psi^{\star} \leftarrow \psi^{\star} \odot_\ell \varphi$}
  }

  \caption{\FOLowerUnits}
  \label{algo:FOLU}
\end{algorithm}



\clearpage
