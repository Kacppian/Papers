\section{The Resolution Calculus}

We assume that there are infinitely many variable symbols (e.g. $X$, $Y$, $Z$, $X_1$, $X_2$, \ldots), constant symbols (e.g. $a$, $b$, $c$, $a_1$, $a_2$, \ldots), function symbols of every arity (e.g $f$, $g$, $f_1$, $f_2$, \ldots) and predicate symbols of every arity (e.g. $p$, $q$, $p_1$, $p_2$,\ldots). A \emph{term} is any variable, constant or the application of an $n$-ary function symbol to $n$ terms.
An \emph{atomic formula} (\emph{atom}) is the application of an $n$-ary predicate symbol to $n$ terms. A \emph{literal} is an atom or the negation of an atom. The
\emph{complement} of a literal $\ell$ is denoted $\dual{\ell}$ (i.e. for any atom $p$,
$\dual{p} = \neg p$ and $\dual{\neg p} = p$). The set of all literals is denoted $\mathcal{L}$. A
\emph{clause} is a multiset of literals. $\bot$ denotes the \emph{empty clause}. A \emph{unit clause} is a clause with a single literal. Sequent notation is used for clauses (i.e. $p_1,\ldots,p_n \seq q_1,\ldots, q_m$ denotes the clause $\{ \neg p_1,\ldots, \neg p_n, q_1, \ldots, q_m \}$).
$\freevar{t}$ (resp. $\freevar{\ell}$, $\freevar{\clause}$) denotes the set of variables in the term $t$ (resp. in the literal $\ell$ and in the clause $\clause$).
A \emph{substitution} $\{ X_1\backslash t_1, X_2 \backslash t_2, \ldots \}$ is a mapping from variables $\{ X_1, X_2, \ldots \}$ to, respectively, terms $\{t_1, t_2, \ldots \}$. The application of a substitution $\sigma$ to a term $t$, a literal $\ell$ or a clause $\clause$ results in, respectively, the term $t \sigma$, the literal $\ell \sigma$ or the clause $\clause \sigma$, obtained from $t$, $\ell$ and $\clause$ by replacing all occurrences of the variables in $\sigma$ by the corresponding terms in $\sigma$. The set of all substitutions is denoted $\mathcal{S}$. A \emph{unifier} of a set of literals is a substitution that makes all literals in the set equal.
A \emph{resolution proof} is a directed acyclic graph of clauses where the edges correspond to the inference rules of resolution and contraction (as explained in detail in Definition \ref{def:proof}). A \emph{resolution refutation} is a resolution proof with root $\bot$.


\begin{definition}[First-Order Resolution Proof] 
\label{def:proof} \hfill \\
A directed acyclic graph $\langle V, E, \clause \rangle$, where $V$ is a set of nodes and $E$ is a
set of edges labeled by literals and substitutions (i.e. $E \subset V \times 2^{\mathcal{L}} \times \mathcal{S} \times V$ and $\n_1
\xrightarrow[\sigma]{\ell} \n_2$ denotes an edge from node $\n_1$ to node $\n_2$ labeled by the literal $\ell$ and the substitution $\sigma$), is a
proof of a clause $\clause$ iff it is inductively constructible according to the following cases:
%
\begin{itemize}
  \item \textbf{Axiom:} If $\Gamma$ is a clause, $\axiom{\Gamma}$ denotes some proof $\langle \{ \n \}, \varnothing,
    \Gamma \rangle$, where $\n$ is a new (axiom) node.
  \item \textbf{Resolution:} If $\psi_L$ is a proof $\langle V_L, E_L, \clause_L \rangle$ with $\ell_L \in \clause_L$ and
    $\psi_R$ is a proof $\langle V_R, E_R, \clause_R \rangle$ with $\ell_R \in \clause_R$, and 
    $\sigma_L$ and $\sigma_R$ are substitutions such that
    $\ell_L \sigma_L = \dual{\ell_R} \sigma_R$ and
    $\freevar{\left( \clause_L \setminus \left\{ \ell_L \right\} \right) \sigma_L} \cap 
     \freevar{\left( \clause_R
                    \setminus \left\{ \ell_R \right\} \right) \sigma_R} = \emptyset$, 
    then
    $\psi_L \res{\ell_L}{\sigma_L}{\ell_R}{\sigma_R} \psi_R$ denotes a proof $\langle V, E, \Gamma \rangle$ s.t.
    \begin{align*}
      V &= V_L \cup V_R \cup \{\n \} \\
      E &= E_L \cup E_R \cup
                    \left\{ \raiz{\psi_L} \xrightarrow[\sigma_L]{\{\ell_L\} } \n, 
                            \raiz{\psi_R} \xrightarrow[\sigma_R]{\{\ell_R\} } \n \right\} \\
     \Gamma &= \left( \clause_L \setminus \left\{ \ell_L \right\} \right) \sigma_L \cup \left( \clause_R
                    \setminus \left\{ \ell_R \right\} \right) \sigma_R 
    \end{align*}
    where $\n$ is a new (resolution) node and $\raiz{\varphi}$ denotes the root node of $\varphi$. The \emph{resolved atom} $\ell$ is such that $\ell = \ell_L \sigma_L = \dual{\ell_R} \sigma_R$ or $\ell = \dual{\ell_L} \sigma_L = \ell_R \sigma_R$.
  \item \textbf{Contraction:} If $\psi'$ is a proof $\langle V', E', \clause' \rangle$ and $\sigma$ is a unifier of $\{\ell_1, \ldots \ell_n\}$ with $\{\ell_1, \ldots \ell_n\} \subseteq \clause'$, then $\con{\psi}{\{\ell_1, \ldots \ell_n\}}{\sigma}$ denotes a proof $\langle V, E, \Gamma \rangle$ s.t.
    \begin{align*}
      V &= V' \cup \{\n \} \\
      E &= E' \cup \{ \raiz{\psi'} \xrightarrow[\sigma]{\{\ell_1, \ldots \ell_n\}} \n \} \\
     \Gamma &= (\clause' \setminus \{ \ell_1, \ldots \ell_n \} ) \sigma \cup \{ \ell \}
    \end{align*}
    where $\n$ is a new (contraction) node, $\ell = \ell_k \sigma$ (for any $k \in \{1,\ldots, n\}$) and $\raiz{\varphi}$ denotes the root node of $\varphi$.
  \qed
\end{itemize}
\end{definition}


\noindent
The resolution and contraction (factoring) rules described above are the standard rules of the resolution calculus, except for the fact that we do not require resolution to use most general unifiers. The presentation of the resolution rule here uses two substitutions, in order to explicitly handle the necessary renaming of variables, which is often left implicit in other presentations of resolution.

When we write $\psi_L \res{\ell_L}{}{\ell_R}{} \psi_R$, we assume that the omitted substitutions are such that the resolved atom is most general. 
We write $\con{\psi}{}{}$ for an arbitrary maximal contraction, and $\con{\psi}{}{\sigma}$ for a (pseudo-)contraction that does merge no literals but merely applies the substitution $\sigma$. 
When the literals and substitutions are irrelevant or clear from the context, we may write simply $\psi_L \res{}{}{}{} \psi_R$ instead of $\psi_L \res{\ell_L}{\sigma_L}{\ell_R}{\sigma_R} \psi_R$.
The $\res{}{}{}{}$ operator is assumed to be left-associative. 
In the propositional case, we omit contractions (treating clauses as sets instead of multisets) and $\psi_L \res{\ell}{\emptyset}{\dual{\ell}}{\emptyset} \psi_R$ is abbreviated by $\psi_L \odot_{\ell} \psi_R$.

If $\psi = \varphi_L \odot \varphi_R$ or $\psi = \con{\varphi}{}{}$, then $\varphi$, $\varphi_L$ and $\varphi_R$ are \emph{direct
subproofs} of $\psi$ and $\psi$ is a \emph{child} of both $\varphi_L$ and $\varphi_R$. The
transitive closure of the direct subproof relation is the \emph{subproof} relation. A subproof which
has no direct subproof is an \emph{axiom} of the proof.
%
$\Vertices{\psi}$, $\Edges{\psi}$ and $\Conclusion{\psi}$
denote, respectively, the nodes, edges and proved clause (conclusion) of $\psi$. If $\psi$ is a proof ending with a resolution node, then $\psi_L$ and $\psi_R$ denote, respectively, the left and right premises of $\psi$.
