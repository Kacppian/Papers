\section{Conclusion}
\label{sec:conclusion}

The problem of compression proofs in space has been reduced to finding strategies in a Pebbling Game, for which finding the optimal strategy is known to be NP-complete.
Therefore, two algorithms for compressing proofs in space have been conceived. The experimental evaluation clearly shows that the so-called Bottom-Up algorithms are faster and compress more than the more natural, straightforward and simple Top-Down algorithms. Both kinds of algorithms are parametrized by a heuristic function for selecting nodes. The best performances are achieved with the simplest heuristics (i.e. Last Child and Number of Children). More sophisticated heuristics provided little extra compression but cost a high price in execution time. Future work could investigate heuristics that take advantage of the particular shape of proofs generated by analysis of conflict graphs.
Furthermore, the methods could be implemented directly into a SAT- or SMT-solver to provide proofs with small space right away.

\vspace{-5pt}
\paragraph{Acknowledgments:} We would like to thank Armin Biere for clarifying why resolution chains are not left-associative in the TraceCheck proof format.

\vspace{-5pt}

